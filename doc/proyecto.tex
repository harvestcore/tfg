\documentclass[11pt, a4paper]{book}
\usepackage{listings}
\usepackage[utf8]{inputenc}
\usepackage[english,spanish]{babel}
\RequirePackage{verbatim}
\usepackage{fancyhdr}
\usepackage{graphicx}
\usepackage{afterpage}
\usepackage{pdfpages}
\usepackage{longtable}

\usepackage{adjustbox}

\usepackage[pdfborder={000}]{hyperref} %referencia

% ********************************************************************
% Re-usable information
% ********************************************************************
\newcommand{\myTitle}{Título del proyecto\xspace}
\newcommand{\myDegree}{Grado en Ingeniería Informática\xspace}
\newcommand{\myName}{Ángel Gómez Martín\xspace}
\newcommand{\myProf}{Juan Julián Merelo Guervos\xspace}
\newcommand{\myFaculty}{Escuela Técnica Superior de Ingenierías Informática y de Telecomunicación\xspace}
\newcommand{\myFacultyShort}{E.T.S. de Ingenierías Informática y de
Telecomunicación\xspace}
\newcommand{\myDepartment}{Departamento de Arquitectura y Tecnología de Computadores\xspace}
\newcommand{\myUni}{\protect{Universidad de Granada}\xspace}
\newcommand{\myLocation}{Granada\xspace}
\newcommand{\myTime}{\today\xspace}
\newcommand{\myVersion}{Version 0.1\xspace}

\usepackage{hyperref}
\usepackage{url}
\usepackage{colortbl,longtable}
%\makeindex
%\usepackage[style=long, cols=2,border=plain,toc=true,number=none]{glossary}
%\makeglossary


\setlength{\headheight}{1.5\headheight}

\definecolor{gray97}{gray}{.97}
\definecolor{gray75}{gray}{.75}
\definecolor{gray45}{gray}{.45}
\definecolor{gray30}{gray}{.94}

\lstset{ frame=Ltb,
     framerule=0.5pt,
     aboveskip=0.5cm,
     framextopmargin=3pt,
     framexbottommargin=3pt,
     framexleftmargin=0.1cm,
     framesep=0pt,
     rulesep=.4pt,
     backgroundcolor=\color{gray97},
     rulesepcolor=\color{black},
     %
     stringstyle=\ttfamily,
     showstringspaces = false,
     basicstyle=\scriptsize\ttfamily,
     commentstyle=\color{gray45},
     keywordstyle=\bfseries,
     %
     numbers=left,
     numbersep=6pt,
     numberstyle=\tiny,
     numberfirstline = false,
     breaklines=true,
   }
 
% minimizar fragmentado de listados
\lstnewenvironment{listing}[1][]
   {\lstset{#1}\pagebreak[0]}{\pagebreak[0]}


\newcommand{\bigrule}{\titlerule[0.5mm]}

%Para conseguir que en las páginas en blanco no ponga cabeceras
\makeatletter
\def\clearpage{%
  \ifvmode
    \ifnum \@dbltopnum =\m@ne
      \ifdim \pagetotal <\topskip
        \hbox{}
      \fi
    \fi
  \fi
  \newpage
  \thispagestyle{empty}
  \write\m@ne{}
  \vbox{}
  \penalty -\@Mi
}
\makeatother

\begin{document}
	\begin{titlepage}
 
 
\newlength{\centeroffset}
\setlength{\centeroffset}{-0.5\oddsidemargin}
\addtolength{\centeroffset}{0.5\evensidemargin}
\thispagestyle{empty}

\noindent\hspace*{\centeroffset}\begin{minipage}{\textwidth}

\centering
\includegraphics[width=0.9\textwidth]{imagenes/logo_ugr.jpg}\\[1.4cm]

\textsc{ \Large TRABAJO FIN DE GRADO\\[0.2cm]}
\textsc{GRADO EN INGENIERÍA INFORMÁTICA}\\[1cm]
% Upper part of the page
% 
% Title
{\Huge\bfseries IPManager\\}
\noindent\rule[-1ex]{\textwidth}{3pt}\\[3.5ex]
{\large\bfseries Infraestructura bajo demanda}
\end{minipage}

\vspace{2.5cm}
\noindent\hspace*{\centeroffset}\begin{minipage}{\textwidth}
\centering

\textbf{Autor}\\{Ángel Gómez Martín}\\[2.5ex]
\textbf{Director}\\Juan Julián Merelo Guervós\\[2cm]
\includegraphics[width=0.3\textwidth]{imagenes/etsiit_logo.png}\\[0.1cm]
\textsc{Escuela Técnica Superior de Ingenierías Informática y de Telecomunicación}\\
\textsc{---}\\
Granada, Julio de 2020
\end{minipage}
%\addtolength{\textwidth}{\centeroffset}
%\vspace{\stretch{2}}
\end{titlepage}



	\begin{titlepage}
 
 
\setlength{\centeroffset}{-0.5\oddsidemargin}
\addtolength{\centeroffset}{0.5\evensidemargin}
\thispagestyle{empty}

\noindent\hspace*{\centeroffset}\begin{minipage}{\textwidth}

\centering


\vspace{3.3cm}

% Title

{\Huge\bfseries IPManager}
\noindent\rule[-1ex]{\textwidth}{3pt}\\[3.5ex]
{\large\bfseries Infraestructura bajo demanda\\[4cm]}
\end{minipage}

\vspace{2.5cm}
\noindent\hspace*{\centeroffset}\begin{minipage}{\textwidth}
\centering

\textbf{Autor}\\ {Ángel Gómez Martín}\\[2.5ex]
\textbf{Director}\\
{Juan Julián Merelo Guervós}\\[2cm]
%\includegraphics[width=0.15\textwidth]{imagenes/tstc.png}\\[0.1cm]
%\textsc{Departamento de Teoría de la Señal, Telemática y Comunicaciones}\\
%\textsc{---}\\
Granada, Julio de 2020
\end{minipage}
%\addtolength{\textwidth}{\centeroffset}
\vspace{\stretch{2}}

 
\end{titlepage}




\cleardoublepage

\chapter*{}
\thispagestyle{empty}

\begin{center}
{\large\bfseries Título del Proyecto: Subtítulo del proyecto}\\
\end{center}
\begin{center}
Ángel Gómez Martín
\end{center}

\noindent{\textbf{Palabras clave}: palabra\_clave1, palabra\_clave2, palabra\_clave3, ......}\\

\vspace{0.7cm}
\noindent{\textbf{Resumen}}\\

Se pretende desarrollar una aplicación web que permita al usuario crear y desplegar una infraestructura bajo demanda desde un puesto centralizado para integrar ambos procedimientos en un mismo proceso. Debido a la gran cantidad de tiempo que consume sendas tareas, la propuesta intenta que se agilice al máximo posible, además de una manera sencilla, el desarrollo de las mismas.

\cleardoublepage
\thispagestyle{empty}

\begin{center}
{\large\bfseries Project Title: Project Subtitle}\\
\end{center}
\begin{center}
Ángel Gómez Martín
\end{center}

\noindent{\textbf{Keywords}: Keyword1, Keyword2, Keyword3, ....}\\

\vspace{0.7cm}
\noindent{\textbf{Abstract}}\\

The purpose of this project is to develop a web application that allows the user to create and deploy an infrastructure on demand from a centralized system, in order to integrate both procedures in the same process. Due to the large amount of time consumed by each task, the proposal tries to speed up those as much as possible in a simple way.












\chapter*{}
\thispagestyle{empty}

\noindent\rule[-1ex]{\textwidth}{2pt}\\[4.5ex]

Yo, \textbf{Ángel Gómez Martín}, alumno de la titulación TITULACIÓN de la \textbf{Escuela Técnica Superior de Ingenierías Informática y de Telecomunicación de la Universidad de Granada}, con DNI XXXXXXXXX, autorizo la ubicación de la siguiente copia de mi Trabajo Fin de Grado en la biblioteca del centro para que pueda ser consultada por las personas que lo deseen.

\vspace{6cm}

\noindent {Fdo: Ángel Gómez Martín}

\vspace{2cm}












\chapter*{}
\thispagestyle{empty}

\noindent\rule[-1ex]{\textwidth}{2pt}\\[4.5ex]

D. \textbf{Juan Julián Merelo Guervos}, Profesor del Área de XXXX del Departamento de Arquitectura y Tecnología de Computadores de la Universidad de Granada.

\vspace{0.5cm}

\textbf{Informa:}

\vspace{0.5cm}

Que el presente trabajo, titulado \textit{\textbf{Título del proyecto, Subtítulo del proyecto}},
ha sido realizado bajo su supervisión por \textbf{Ángel Gómez Martín}, y autoriza la defensa de dicho trabajo ante el tribunal que corresponda.

\vspace{0.5cm}

Y para que conste, expide y firma el presente informe en Granada a X de mes de 2020.

\vspace{1cm}

\textbf{El director:}

\vspace{5cm}

\noindent {Fdo: Juan Julián Merelo Guervos}








\chapter*{Agradecimientos}
\thispagestyle{empty}

       \vspace{1cm}


Poner aquí agradecimientos...



	\tableofcontents
	%\listoffigures
	%\listoftables

	\setlength{\parskip}{5pt}

	\chapter{Introducción}


\textbf{Ámbito}
\vspace{0,2cm}

El sistema está centrado en empresas de pequeño y mediano tamaño, donde el número de equipos a aprovisionar es considerable y creciente así como el número de servicios a utilizar está también en incremento.


\vspace{1cm}


\textbf{Extensión}
\vspace{0,2cm}

Se pretende que el sistema pueda realizar las siguientes tareas:

- Despliegue de servicios on-premise y en la nube.
- Aprovisionamiento de sistemas.

Además, dispondría de una interfaz de usuario gráfica así como un cliente de línea de comandos.


\vspace{1cm}


\textbf{Notas}
\vspace{0,2cm}

- Backend y frontend.

- Probablemente en:

\hspace{1cm}- Python: Hay bastantes librerias y lo conozco bien.

\hspace{1cm}- Go: No hay tantas librerias y me gustaría aprender a usarlo.

\hspace{1cm}- Angular: Lo conozco bastante bien.

- Crear scripts de aprovisionamiento y ejecutarlos.

- Poder crear imagenes de docker / ejecutarlas.

- Despliegue en la nube y on-premise.

- Mongo como db.

- Guardar configuraciones (?)

- Multiclient (?)

- Tests de back y front.

	
	\chapter{Análisis}

\section{Descripción de los actores}

En este sistema hay dos actores:

\textbf{Usuario regular}. Este actor puede realizar todas las acciones disponibles en el backend. Aun teniendo conocimientos en la administración de sistemas, no se le permite realizar acciones de este tipo sobre los clientes y usuarios del sistema.

\textbf{Admistrador}. Este actor puede realizar todas las acciones que se le permiten a un usuario regular y además puede realizar tareas de administración en los clientes y usuarios del sistema.

\section{Requisitos del sistema}

\subsection{Requisitos funcionales}
\begin{itemize}
	\item \textbf{R.F. 1} Se distinguirán los clientes por medio de un subdominio en la \textit{URL}.
	\item \textbf{R.F. 2} El sistema tendrá un sistema de autenticación.
	\item \textbf{R.F. 3} La visualización de datos en el frontend deberá ser en forma de listados.
	\item \textbf{R.F. 4} El backend proveerá una \textit{API} capaz de funcionar sin la necesidad de un frontend.
	\item \textbf{R.F. 5} El sistema permitirá aprovisionar máquinas.
	\item \textbf{R.F. 6} En el sistema podrá haber distintos clientes.
	\item \textbf{R.F. 7} El sistema permitirá la administración de clientes.
	\item \textbf{R.F. 8} En cada cliente podrá haber diferentes usuarios.
	\item \textbf{R.F. 9} El sistema permitirá la administración de usuarios.
	\item \textbf{R.F. 10} Los usuarios podrán ser de tipo administrador o usuario común.
	\item \textbf{R.F. 11} Un usuario administrador podrá crear usuarios comunes.
\end{itemize}


\subsection{Requisitos no funcionales}
\begin{itemize}
	\item \textbf{R.N.F. 1} La autenticación del usuario será mediante \textit{JWT}.
	\item \textbf{R.N.F. 2} El puesto centralizado estará compuesto por un backend y un frontend.
	\item \textbf{R.N.F. 3} El despliegue de servicios se hará mediante contenedores Docker.
	\item \textbf{R.N.F. 4} El frontend tendrá una interfaz sencilla.
	\item \textbf{R.N.F. 5} El sistema funcionará para sistemas basados en \textit{GNU Linux}.
	\item \textbf{R.N.F. 6} El sistema deberá ser escalable.
	\item \textbf{R.N.F. 7} La interfaz de usuario del sistema será mediante una aplicación web.
\end{itemize}

\subsection{Requisitos de información}
\begin{itemize}
	\item \textbf{R.I. 1} Se almacenarán las diferentes configuraciones de aprovisionamiento.
	\item \textbf{R.I. 2} El sistema almacenará los detalles de los sistemas que aprovisiona.
	\item \textbf{R.I. 3} El sistema almacenará para cada cliente un identificador único, un dominio y el nombre de la base de datos correspondiente a ese cliente.
	\item \textbf{R.I. 4} El sistema almacenará para todo cliente y usuario si ha sido eliminado, no borrando sus datos tras eliminarlo.
	\item \textbf{R.I. 5} El sistema almacenará para todo usuario registrado un identificador único, el tipo de usuario (administrador o regular), el nombre y apellido del usuario, un email, un nombre de usuario y una contraseña almacenada en un formato seguro.
	\item \textbf{R.I. 6} El sistema almacenará para cada máquina un identificador único, un nombre de la máquina, una descripción de la máquina, su dirección IPv4 y un conjunto de identificadores de scripts asociados a esa máquina.
	\item \textbf{R.I. 7} El sistema almacenará para cada script de aprovisionamiento un identificador único, un nombre y el script en sí.
	\item \textbf{R.I. 8} El sistema almacenará para cada grupo de hosts un identificador único, un nombre y el conjunto de direcciones IP asociadas.
	\item \textbf{R.I. 9} El sistema almacenará para cada máquina un identificador único, un nombre, una descripción, un tipo de máquina, dirección IPv4 e IPv6, dirección MAC, máscara de red, dirección broadcast y dirección de red.
\end{itemize}



\section{Modelo de negocio y presupuesto}

Aunque la solución propuesta se caracteriza por ser software libre el coste de desarrollo y de implantación nunca es cero.

En el momento de comenzar el desarrollo los fondos son escasos, por lo que el modelo de negocio inicial estaría centrado en obtener los ingresos mínimos que permitan continuar con el proyecto. Una vez superado este primer obstáculo y el software se encuentre más asentado se cambiaría el modelo de financiación para poder obtener mayores beneficios y valor de mercado.

Por tanto se crearía una \textit{Sociedad Limitada de Nueva Empresa}, la cual permite crear una pequeña empresa con pocos recursos iniciales y ofrece ciertas ventajas en este tipo de proyectos:

\begin{itemize}
	\item Rápida constitución.
	\item No es necesario un registro de socios.
	\item Se pueden aplazar deudas del impuesto de sociedades y no existe obligación de realizar pagos fraccionados de este.
	\item El cambio de denominación social es gratuito temporalmente.
	\item Se permite el aplazamiento y fraccionado de retenciones del \textit{IRPF}.
\end{itemize}

El salario medio (dato de 2020) en el sector de la Información y Telecomunicaciones se sitúa en torno a los 34000 euros brutos anuales, lo que supondría aproximadamente unos 2800 euros brutos mensuales, una cifra que en el momento de creación de la empresa es inviable. Una consulta al Instituto Nacional de Estadística nos revela que el gasto medio anual por persona (datos de 2018) es de 12000 euros, lo que se traduce en unos 1000 euros mensuales. A fin de reducir los gastos al máximo y haciendo un promedio a la baja de las dos cifras manejadas anteriormente supondremos un sueldo bruto por trabajador de 1500 euros mensuales (18000 euros anuales).

Otros gastos importantes a tener en cuenta son:
\begin{itemize}
	\item El capital inicial a aportar en el momento de creación de la empresa (el mínimo son 3000 euros).
	\item La cuota de autónomos, que en 2020 es de 286,15 euros y que debe pagarse mensualmente, lo que asciende a 3433,80 euros al año.
	\item Los gastos burocráticos, que ascienden aproximadamente a 250 euros.
	\item Local en el que desarrollar la actividad laboral y gastos derivados, aproximadamente 600 euros. 
\end{itemize}

En cuanto a gastos derivados del desarrollo del software se encuentran algunos bienes y servicios como podrían ser repositorios para el almacenamiento del código, integración continua, \textit{Cloud Computing}, adquisición de equipos informáticos y de telecomunicaciones, etc. Muchos de estos servicios tienen versiones gratuitas, lo que ayudaría en el desarrollo inicial del negocio, pero en el momento que se integren nuevos componentes en el equipo puede ser necesario adquirir planes que ofrezcan mejores características.

\bigskip
Las siguientes tablas resumen los gastos iniciales y anuales a afrontar:

\begin{table}[!h]
	\centering
	\begin{adjustbox}{width=1\textwidth}
	\begin{tabular}{|l|l|l|l|}
		\hline
		\textbf{Concepto} & \textbf{Euros/Ud.} & \textbf{Cantidad} & \textbf{Total (en Euros)} \\ \hline
		Capital inicial SLNE & 3000 & 1 & 3000 \\ \hline
		Trámites & 250 & 1 & 250 \\ \hline
		Equipos informáticos & 2500 & 1 & 2500 \\ \hline
		&  &  &  \\
		&  & \textbf{Total} & 5750 \\ \hline
	\end{tabular}
	\end{adjustbox}
\end{table}

\begin{table}[!h]
	\centering
	\begin{adjustbox}{width=1\textwidth}
	\begin{tabular}{|l|l|l|l|}
		\hline
		\textbf{Concepto} & \textbf{Euros/Ud.} & \textbf{Cantidad} & \textbf{Total (en Euros)} \\ \hline
		Salario & 1500 & 12 & 18000 \\ \hline
		Cuota autónomos & 286,15 & 12 & 3433,80 \\ \hline
		Local y derivados & 600 & 12 & 7200 \\ \hline
		&  &  &  \\
		&  & \textbf{Total} & 28633,8 \\ \hline
	\end{tabular}
	\end{adjustbox}
\end{table}

Como se observa el primer año de vida de la empresa costaría aproximadamente 35000 euros, esto teniendo en cuenta que solo tendría un empleado inicial, ya que en el momento de ampliar el equipo cada nuevo integrante implicaría unos 22000 euros al año más.

Aunque se quiere ofrecer una solución de software libre, para sufragar estos gastos se implementaría un sistema de suscripciones mensuales o anuales, en las que se ofrezca soporte personalizado y funcionalidades adicionales o personalizadas.
	
	\chapter{Estado del arte}

En este capítulo repasan de las diferentes soluciones que existen actualmente al problema expuesto. Cada una de ellas destaca sobre las demás en un aspecto u otro, pero ninguna llega a aunar todas las características deseadas. Tras eso se hace una crítica a las soluciones vistas y se comenta por qué no son ideales.

\section{Soluciones actuales}

En el ámbito de los despliegues y automatización hay una gran cantidad de software que en mayor o menor medida permiten realizar estas tareas. Con características similares a las de este proyecto destacan tres y son las siguientes:

\begin{itemize}
	\item \textbf{GECOS}. \textit{Guadalinex Escritorio Corporativo Estándar} es un proyecto de la Junta de Andalucía que ofrece una distribución de \textit{GNU Linux} centrada en la administración de sistemas. Este ecosistema, que está basado totalmente en software libre, actúa como centro de control, que permite el despliegue de equipos, su soporte y administración. Además cuenta con un repositorio de software para proveer a las máquinas que se aprovisionan. Utiliza herramientas y tecnologías como \textit{MongoDB} para la gestión de los datos, \textit{Chef} para el aprovisionamiento o \textit{Celery} para el manejo de las colas de tareas.
	
	Aunque no se trata de un sistema altamente intrusivo para las máquinas que gestiona, este sí requiere que se instale una imagen en la máquina que se vaya a utilizar como maestra. De este modo sólo un equipo es el que tiene acceso a la administración del resto. La gestión de los sistemas se hace desde una interfaz web y permite organizar los equipos según los criterios que se desee e identifica a cada uno de ellos mediante certificados digitales únicos.
		
	\item \textbf{Ansible Tower}. Es una solución de \textit{Red Hat} para la administración de ecosistemas de ámbito empresarial ya que se centra en multitud de sistemas operativos, servidores, infraestructuras virtuales y redes. Esta gestión se realiza desde una interfaz web o se puede integrar con cualquier otro sistema mediante una \textit{API REST}.
	
	La principal característica es que todos los procesos se pueden realizar de forma muy visual y con poca configuración por parte del administrador. Se centra principalmente en el aprovisionado de sistemas mediante \textit{Ansible}, aunque también permite monitorizar estas máquinas y gestionar permisos a los diferentes usuarios.
		
	\item \textbf{Portainer}. Se trata de una herramienta de gestión y administración de contenedores que amplía las funcionalidades que ofrece \textit{Kinematic}, la \textit{GUI} oficial a la que \textit{Docker} da soporte. Al igual que las anteriores soluciones \textit{Portainer} ofrece sus capacidades a través de una interfaz web y es la más liviana de todas ya que se puede ejecutar fácilmente en un contenedor.
	
	Permite administrar todos los aspectos de \textit{Docker} como son las imágenes, contenedores, redes y volúmenes entre otros. Además la interacción con todos estos aspectos es muy sencilla. Se trata también de un proyecto \textit{open source}, por lo que es gratuíto.
\end{itemize}


\section{Crítica}

Si bien todas estas herramientas mencionadas permiten realizar las funcionalidades que se desean ninguna de ellas las aúna todas. En primer lugar \textit{GECOS} ofrece una herramienta de administración de sistemas y de aprovisionado, pero no ofrece ningún tipo de despliegue de servicios. \textit{Red Hat} hace uso de su herramienta estrella en el aprovisionado de sistemas y la lleva al siguiente nivel con \textit{Ansible Tower}, pero al igual que con \textit{GECOS} no ofrece nada relevante en cuando a servicios en contenedores. Por último \textit{Portainer} provee de una excelente solución para este problema, pero no para el aprovisionado y gestión de configuraciones.

Otro aspecto importante es el económico. \textit{GECOS} y \textit{Portainer} son sistemas completamente gratuítos, por lo que serían una opción ideal. En el caso del segundo, \textit{Portainer} ofrece planes superiores con mayor soporte y funcionalidades avanzadas, pero la versión principal es gratuíta. En cambio \textit{Ansible Tower} es de pago, teniendo la suscripción más básica un coste aproximado anual de 5000 dólares anuales y la más avanzada de unos 14000 dólares. Si bien estos precios pueden ser asequibles para una gran organización, para una pequeña o mediana empresa pueden ser inasumibles.

\pagebreak
	
	\chapter{Planificación}

En este capítulo se expone la metodología de desarrollo que se ha empleado en este proyecto, la cual es una mezcla entre SCRUM y Kanban y las razones por las que se ha elegido este sistema frente a las metodologías clásicas. También se indican todas las tareas que se han realizado en cada \textit{sprint}, los cuales corresponden a cada uno de los meses en los que ha transcurrido el desarrollo.

\section{Metodología de desarrollo}

Podría definirse una metodología de desarrollo como el proceso disciplinado de desarrollo de software con el fin de hacerlo más eficiente. En la actualidad existen muchas de estas metodologías que han surgido a lo largo del tiempo mejorándose unas a otras teniendo en cuenta factores como los costes, planificación, calidad y las dificultades asociadas al desarrollo de un software.

Aunque las bases esenciales no difieren entre metodologías sí hay diferencias en los ámbitos en los que se centran principalmente. Por este motivo voy a utilizar una mezcla de dos metodologías ágiles: \textit{SCRUM} y \textit{Kanban}.

Lo interesante de Scrum es la forma de dividir el proceso de desarrollo del software. Para ello se usan \textit{sprints}, los cuales son un periodo de tiempo (variables) en el que se planifican tareas, se desarrollan y luego se entregan de forma funcional. Esto permite entregas paulatinas, un \textit{feedback} continuo y un desarrollo más dinámico por parte de todos los implicados.

Agrupa todas las buenas prácticas de las metodologías ágiles, y si bien en mi caso estoy algo más limitado al ser \textit{Product Owner}, \textit{Scrum Master} y desarrollador al mismo tiempo, la organización es muy atractiva en este tipo de proyectos. La división por \textit{sprints} hace que cada mes (tiempo que estimo suficiente para un \textit{sprint} completo) se tenga un conjunto de funcionalidades nuevas. Por otro lado, las reuniones diarias son conmigo mismo (y en ocasiones con el tutor, las cuales pueden solucionar dudas o problemas que surgen), lo que permite que la organización sea mejor y sepa qué hacer cada día.

\textit{Kanban} por su parte es una metodología que utiliza tarjetas para simbolizar las tareas que se tienen que realizar en el desarrollo de un software. Estas tarjetas se utilizan en un tablero dividido por columnas, las cuales simbolizan tareas que no se han empezado aún, tareas que estan en progreso, ya terminadas, etc.

Pienso que la organización visual de las tareas es bastante provechosa, pues dicha visualización permite saber las tareas que son más urgentes o el estado en el que se encuentran. \textit{GitHub} cuenta con una herramienta llamada \textit{Proyectos} que ofrece un tablero \textit{Kanban} para la organización de tareas y además integración con las \textit{issues} y \textit{pull requests} que se van creando en el proyecto, algo que pienso que es beneficioso para el proyecto y que se ha usado durante el desarrollo del mismo.

Las metodologías tradicionales surgieron cuando aparecieron los primeros sistemas software y están caracterizadas por tener una estructura lineal, en la que al principio se acuerdan las características que debe tener el software y no se modifican durante el desarrollo del sistema, y finalmente se entrega el producto sin dar lugar a cambios. Por lo general se centran en la documentación exhaustiva del proyecto y suelen ser llevadas a cabo por equipos compuestos por multitud de desarrolladores. Además, durante el desarrollo no se tiene una especial comunicación con el cliente.

Aunque tienen algunas ventajas, como tener los objetivos claros desde el primer momento o tener un seguimiento continuo, las desventajas hacen que no tenga cabida este tipo de metodología en el proyecto. Estas son los costes elevados, que no se permitan cambios durante el desarrollo y que la entrega del producto se haga al final del desarrollo, entre otros.


\section{Temporización}

Como se indicaba en la sección anterior los \textit{sprints} planeados tienen una duración aproximada de un mes. Comencé el desarrollo del proyecto a comienzos del mes de febrero y lo he terminado a finales de junio, por lo que se han realizado cinco \textit{sprints}. En total se han creado y finalizado 61 \textit{issues} que se han ido repartiendo por los hitos y desarrollando a lo largo de todo este tiempo. 

Al comienzo del desarrollo se definieron tres hitos que corresponden con las tres partes principales de este proyecto. Son: desarrollo del backend, desarrollo del frontend y documentación.

\bigskip
El trabajo realizado dividido por \textit{sprints} ha sido el siguiente:

\subsection{Febrero}
Comienzo de la investigación de herramientas y tecnologías, además de algunas tareas de documentación. Creación del repositorio en \textit{GitHub}, del proyecto del backend y pipelines en \textit{GitHub Actions}. Clase \textit{Item}, \textit{Client} y sus servicios. Creación del \textit{login} y del despliegue de servicios. Clase \textit{DockerEngine}.

\bigskip
\textbf{Issues:} \href{https://github.com/harvestcore/tfg/issues/1}{DEV-1}, \href{https://github.com/harvestcore/tfg/issues/3}{DEV-3}, \href{https://github.com/harvestcore/tfg/issues/4}{DEV-4}, \href{https://github.com/harvestcore/tfg/issues/6}{DEV-6}, \href{https://github.com/harvestcore/tfg/issues/11}{DEV-11}, \href{https://github.com/harvestcore/tfg/issues/12}{DEV-12}, \href{https://github.com/harvestcore/tfg/issues/13}{DEV-13}, \href{https://github.com/harvestcore/tfg/issues/17}{DEV-17}, \href{https://github.com/harvestcore/tfg/issues/25}{DEV-25}, \href{https://github.com/harvestcore/tfg/issues/27}{DEV-27} y \href{https://github.com/harvestcore/tfg/issues/29}{DEV-29}.


\subsection{Marzo}
Aprovisionado de máquinas. \textit{Multiclient} en el backend. Arreglo de algunos \textit{bugs}. \textit{Dockerizado} del backend y creación de \textit{pipelines}.

\bigskip
\textbf{Issues:} \href{https://github.com/harvestcore/tfg/issues/22}{DEV-22},  \href{https://github.com/harvestcore/tfg/issues/32}{DEV-32},  \href{https://github.com/harvestcore/tfg/issues/37}{DEV-37},  \href{https://github.com/harvestcore/tfg/issues/40}{DEV-40},  \href{https://github.com/harvestcore/tfg/issues/44}{DEV-44} y  \href{https://github.com/harvestcore/tfg/issues/45}{DEV-45}. 


\subsection{Abril}
Tests adicionales. Creación del \textit{CLI}. Servicios para el estado del servidor y \textit{heartbeat}. Clase \textit{Machine} y servicio para el manejo de máquinas. Test en profundidad del backend y arreglo de \textit{bugs}. Documentación.

\bigskip
\textbf{Issues:} \href{https://github.com/harvestcore/tfg/issues/19}{DEV-19},  \href{https://github.com/harvestcore/tfg/issues/24}{DEV-24},  \href{https://github.com/harvestcore/tfg/issues/26}{DEV-26},  \href{https://github.com/harvestcore/tfg/issues/35}{DEV-35},  \href{https://github.com/harvestcore/tfg/issues/37}{DEV-37},  \href{https://github.com/harvestcore/tfg/issues/40}{DEV-40},  \href{https://github.com/harvestcore/tfg/issues/43}{DEV-43},  \href{https://github.com/harvestcore/tfg/issues/49}{DEV-49} y  \href{https://github.com/harvestcore/tfg/issues/59}{DEV-59}.


\subsection{Mayo}
\textit{Sprint} dedicado completamente al frontend. Creación del proyecto y \textit{pipelines}. Servicios para comunicación frontend-backend. \textit{Routing}. Diseño. \textit{Login} y \textit{multiclient}. Componentes.

\bigskip
\textbf{Issues:} \href{https://github.com/harvestcore/tfg/issues/33}{DEV-33}, 
\href{https://github.com/harvestcore/tfg/issues/41}{DEV-41}, 
\href{https://github.com/harvestcore/tfg/issues/42}{DEV-42}, 
\href{https://github.com/harvestcore/tfg/issues/68}{DEV-68}, 
\href{https://github.com/harvestcore/tfg/issues/69}{DEV-69}, 
\href{https://github.com/harvestcore/tfg/issues/70}{DEV-70}, 
\href{https://github.com/harvestcore/tfg/issues/73}{DEV-73}, 
\href{https://github.com/harvestcore/tfg/issues/75}{DEV-75}, 
\href{https://github.com/harvestcore/tfg/issues/77}{DEV-77}, 
\href{https://github.com/harvestcore/tfg/issues/78}{DEV-78}, 
\href{https://github.com/harvestcore/tfg/issues/81}{DEV-81}, 
\href{https://github.com/harvestcore/tfg/issues/82}{DEV-82}, 
\href{https://github.com/harvestcore/tfg/issues/83}{DEV-83}, 
\href{https://github.com/harvestcore/tfg/issues/88}{DEV-88}, 
\href{https://github.com/harvestcore/tfg/issues/91}{DEV-91} y 
\href{https://github.com/harvestcore/tfg/issues/94}{DEV-94}.


\subsection{Junio}
Test en profundidad del frontend y arreglo de \textit{bugs}. Documentación y maquetado.

\bigskip
\textbf{Issues:} \href{https://github.com/harvestcore/tfg/issues/20}{DEV-20}, 
\href{https://github.com/harvestcore/tfg/issues/23}{DEV-23}, 
\href{https://github.com/harvestcore/tfg/issues/92}{DEV-92}, 
\href{https://github.com/harvestcore/tfg/issues/100}{DEV-100}, 
\href{https://github.com/harvestcore/tfg/issues/102}{DEV-102}, 
\href{https://github.com/harvestcore/tfg/issues/103}{DEV-103}, 
\href{https://github.com/harvestcore/tfg/issues/104}{DEV-104}, 
\href{https://github.com/harvestcore/tfg/issues/105}{DEV-105}, 
\href{https://github.com/harvestcore/tfg/issues/106}{DEV-106}, 
\href{https://github.com/harvestcore/tfg/issues/107}{DEV-107} y 
\href{https://github.com/harvestcore/tfg/issues/109}{DEV-109}.

	
	\chapter{Herramientas y tecnologías utilizadas}

\section{Arquitectura}

En este aspecto tan esencial la elección está bastante clara y opto por la arquitectura de microservicios. En el proyecto que se desarrolla no tiene sentido un sistema monolítico en el que frontend y backend se encuentren juntos. En este caso las funcionalidades a desarrollar se centran en aspectos muy concretos y los beneficios de este tipo de arquitectura superan a los inconvenientes.

Entre estos beneficios descato en primer lugar la modularidad, ya que el backend sería totalmente independiente del frontend, lo que permite que se puedan desarrollar infinitas interfaces de usuario, cada una adecuada al uso que se vaya a dar del sistema. Por otro lado destaco la escalabilidad, ya que en el caso de necesitar más instancias se pueden desplegar de forma rápida y sencilla; y finalmente la seguridad y aislamiento, ya que cada instancia del sistema no interviene con las demás, aislando en este caso los datos de cada caso de uso.

Tradicionalmente el software ha tenido una arquitectura monolítica en la que todos los servicios y funcionalidades están integrados y programados en un mismo sistema, pero como se indica antes, no tiene sentido en este tipo de proyecto.

Las principales ventajas de los microservicios sobre la arquitectura monolítica son:
\begin{itemize}
	\item Agilidad. No es necesario desarrollar todas las funcionalidades completas, por lo que se pueden reutilizar otros microservicios ya desarrollados para suplir las necesidades actuales.
	\item Modularidad. Cada microservicio es independiente del resto, lo que facilita el desarrollo y el despliegue de estos.
	\item Escalabilidad. Debido a la modularidad de estos la escalabilidad horizontal es asequible y muy beneficiosa.
	\item Seguridad y aislamiento. Cada uno de estos servicios encapsula toda su funcionalidad, quedando aislados del resto. Cualquier tipo de vulnerabilidad de la seguridad queda reducido a una parte del sistema general, evitando así pérdidas de información y posibles fallos a otros microservicios.
\end{itemize}

En cambio este sistema también tiene desventajas. Tener multitud de servicios en ejecución conlleva una configuración y un coste de implantación más alto de lo habitual, además de que no hay una uniformidad a la hora de desempeñar un despliegue. Esto conlleva una complejidad añadida ya que aunque los servicios son mas ligeros y sencillos el sistema que conforman es mucho más complejo. La administración también es más compleja ya que se requieren conocimientos específicos de cada microservicio para este cometido.


\section{Integración continua}

La integración continua es una práctica que consiste en el control de versiones del código que se desarrolla y en la ejecución de pruebas automáticas del mismo de forma periódica con el fin de detectar errores o un mal funcionamiento de una forma rápida. Actualmente es un requisito necesario e indispensable en cualquier tipo de software y debe abordar todos los aspectos del mismo.

GitHub Actions es el sistema elegido para la integración continua. Esto se debe a dos motivos, el primero es que el proyecto se encuentra alojado en GitHub y por otro lado GitHub Actions permite el despliegue de contenedores Docker. Este último aspecto es muy importante en el software que se desarrolla debido a que el backend trabaja con este tipo de tecnología y los tests unitarios deben probar todas estas funcionalidades. Además para las pruebas del resto de funcionalidades el poder ejecutar un contenedor con una base de datos facilita mucho el proceso de integración continua. Así mismo son sistemas que se encuentran perfectamente integrados el uno con el otro.

Existen multitud de servicios para este tipo de pruebas:
\begin{itemize}
	\item Jenkins
	\item Travis CI
	\item Bamboo
	\item GitHub Actions
	\item GitLab CI
	\item Circle CI
\end{itemize}

Todos ellos comparten características y son realmente similares. Algunos como Jenkins permiten la integración de multitud de plugins y otros permiten además despliegues continuos. En el caso de este software la característica que más nos interesa es que sea gratuito y la mayoria de ellos lo son para proyectos de software libre. Esta es otra de las características que hacen que estos sistemas gratuitos tengan tanta popularidad.




\section{Despliegue del software en contenedores}

En este ámbito se va a utilizar Docker para desplegar tanto frontend como backend. Este sistema provee una capa adicional de abstracción y de virtualización de las aplicaciones, lo que nos permite ejecutar un software de manera aislada sin tener que depender de complejas configuraciones de máquinas virtuales o hipervisores. Por otro lado los recursos pueden ser también aislados.

Para la creación de estos contenedores se utilizan los denominados \textit{Dockerfile}, que son archivos de texto plano con las diferentes instrucciones que crearán a voluntad el entorno de ejecución de nuestro software. En el caso de este proyecto se van a crear dos de estos archivos, uno el frontend y otro para el backend, y la creación y despliegue de las imágenes generadas se realizará mediante DockerHub.

Como se ha indicado anteriormente este proceso también se puede automatizar, ya sea con scripts creados a mano o con el uso los hooks de DockerHub, que construyen las imágenes específicas cada vez que se haga un cambio en el código o cada vez que se produzca un evento concreto.

Docker Compose es otra herramienta que facilita aún más este proceso, ya que a partir de un archivo YAML permite crear estos contenedores, configurarlos y conectarlos de una forma muy sencilla. En este proyecto también se incluye uno de estos archivos.

Finalmente existen otras herramientas muy interesantes, como son Kubernetes, OpenShift o Mesos, que nos permiten orquestar y escalar los contenedores según los criterios que se configuren. En este proyecto no se hace uso de ellas pero en el caso de que se quisiera dar un paso más en el despliegue del software podrían ser muy interesantes.


\section{Aprovisionamiento}

Existen muchas herramientas que permiten el aprovisionamiento de sistemas, algunas más centradas en el ámbito Cloud y otras de uso local.

En el caso de este proyecto la opción que mejor encaja es Ansible. Debido a su sencillez de uso y de no requerir un servidor central lo hace ideal para el uso en el backend. Además, ya que se encuentra desarrollado en Python y a que existe un SDK, la integración es inmediata, solo teniendo que desarrollar las funcionalidades que se quieran.

Por otro lado la sencillez de los \textit{Playbooks} lo hace más atractivo aún, ya que la sintaxis de los archivos YAML  es muy sencilla. En cuanto a la conexión mediante SSH solo se requiere que el backend tenga conexión a internet, por lo que no es necesario ningún protocolo o configuración adicional para que funcione.

Algunas otras alternativas son:
\begin{itemize}
	\item Chef. La configuración de las máquinas se hace de forma procedural y se depende de un servidor central que almacene las configuraciones o \textit{recetas}. Además ofrece análisis e informes de las máquinas aprovisionadas.
	\item Puppet. Es un conjunto de herramientas que permiten orquestar y administrar grandes conjuntos de máquinas. Al igual que Chef depende de un servidor central y permite ampliar su funcionalidad a través de módulos.
\end{itemize}



\section{Backend}

\subsection{Lenguaje de programación y framework}

Actualmente existen multitud de lenguajes de programación que podrían usarse sin problema alguno para desarrollar una API de las características que se requieren. Las características deseadas que deberían ofrecernos estos en el caso de este software son sencillas aunque a la par difíciles de encontrar en ocasiones. Algunos de estos lenguajes que se suelen utilizar en el desarrollo de APIs son Java, JavaScript, PHP, Python, Ruby, C\# y Go, entre otros.

\begin{itemize}
	\item Para el manejo de datos se requiere que se pueda conectar a MongoDB y esto lo cumplen los lenguajes mencionados, por lo que todos son buenos candidatos en este caso.
	\item En cuanto al aprovisionamiento se requiere algún tipo de SDK de Ansible. Los lenguajes que soportan esto son Go, Python, Ruby, PHP y JavaScript, mientras que el resto tienen un soporte limitado.
	\item Se debe poder administrar Docker y en este caso, al igual que con MongoDB, todos los lenguajes tienen SDKs disponibles.
\end{itemize}

Por otro lado, debido a la arquitectura del proyecto, no es necesario que el framework elegido tenga la arquitectura MVC ya que de la vista se encarga el frontend. Por este motivo todos aquellos que siguen este modelo quedan descartados. Podrían utilizarse sin problema alguno, pero no tendría mucho sentido ya que no le estaríamos sacando todo el partido posible a los mismos.

Anteriormente se mencionaba como aspecto importante los recursos que tienen estos lenguajes y cierto es que algunos pueden ofrecer más que otros, bien sea porque son más antiguos o bien porque son más usados y la comunidad es mayor. En esto destaca Python, también motivado por opinión personal, ya que existen una infinidad de librerías y recursos para este lenguaje y se puede desarrollar cualquier aplicación de forma sencilla e intuitiva. Además, en el caso del aprovisionamiento, Ansible está programado en Python, por lo que la integración con su SDK sería directa, sin problema alguno.

La ausencia de tipado en Python da una mayor flexibilidad y libertad a la hora de desarrollar, pero puede inducir a errores, por lo que es necesario tener un especial cuidado. Otro aspecto es que se trata de un lenguaje interpretado, algo que agiliza el desarrollo ya que no hay que emplear tiempo extra en el compilado. También tiene una sintaxis muy sencilla que facilita la comprensión del código.

En cuanto al framework, como se indicaba antes no es necesario que disponga de arquitectura MVC, por lo que Django queda descartado. En su defecto se usará Flask junto a otros módulos como Flask-RESTPlus o Marshmallow (usado para la definición de esquemas).

La elección de un lenguaje u otro también depende de los recursos que nos ofrezca, esto es librerías, frameworks y todas aquellas características que hagan destacar un lenguaje sobre otro. También influye la experiencia que se tenga, ya que afrontar un gran proyecto con un lenguaje que nunca has usado puede ser un gran reto. Otros frameworks que se han tenido en cuenta a la hora de esta elección han sido:


\textbf{Java}. Destacan frameworks como Spring y Struts.

\textbf{JavaScript}. Express es el más utilizado y se ejecuta sobre Node.js. Otros ejemplos son Sails o Meteor.

\textbf{PHP}. Los más popular son Slim y Lumen. Ambos son microservicios bastante sencillos con muchas funcionalidades incorporadas, como autenticación, encriptado de datos, eventos y colas.

\textbf{Ruby}. Roda y Sinatra son los más utilizados.

\textbf{C\#}. El más popular es .NET Core, de Microsoft.

\textbf{Go}. Tanto Revel como Gin son los más usados.



\section{Frontend}

En el caso de este proyecto la primera elección que se toma es la de abandonar el \textit{stack} HTML/CSS/JS ya que realmente no ofrece nada novedoso sobre las demás soluciones. Tras esto, elijo Angular.

Angular es un framework que aborda todos los aspectos del desarrollo frontend, desde la parte visual hasta las comunicaciones. Su arquitectura se basa en componentes que se pueden crear y personalizar a voluntad y el lenguaje usado para su desarrollo es TypeScript, lo cual permite un mayor control de los datos que se manejan. Integra además multitud de librerías, como RxJS, para aprovechar sus virtudes. Entre sus principales características destacan el enlace de datos bidireccional (\textit{2-way data-binding}) entre el modelo y la vista, la inyección de servicios y dependencias, que facilita el desarrollo y la comprensión del código, y la validación de datos y mecanismos de seguridad integrados. Por contra la curva de aprendizaje es mas grande, ya que integra multitud de conceptos diferentes que no se contemplan en las demás soluciones.

Por otro lado otras opciones que se han barajado han sido:

\textbf{React}. Tiene un enfoque reactivo y trabaja con un DOM virtual en varias capas, lo que permite que sólo se actualicen aquellas partes de la página que deban actualizarse. También permite la creación y reutilización de componentes personalizados, lo que dota a esta librería de mucha flexibilidad a la hora del desarrollo. Por contra sólo se trata de una librería centrada en la parte visual del frontend, por lo que el manejo de datos entre componentes o cualquier tipo de comunicación externa, como HTTP, quedan a cargo del desarrollador.

\textbf{Vue}. En este caso Vue comparte conceptos de React y de Angular, por lo que sería el punto intermedio entre ambos. Destaca por ser mas liviano que estos y por su simplicidad a la hora del desarrollo, lo que hace que la curva de aprendizaje sea bastante menor. Por el contrario, al igual que con React, las comunicaciones corren a cargo del programador, lo que es un punto en su contra.

En cuanto a estas alternativas, aunque es muy interesante el enfoque que tienen se quedan cortas a la hora de la comunicación con el backend. El desarrollador es el que debe proveer de los métodos de comunicación, lo que requiere más tiempo. En cambio con Angular esos aspectos ya se encuentran integrados, lo que simplifica mucho el proceso. Por otro lado aspectos como el \textit{2-way data-binding} y la validación de datos son también interesantes, ya que aportan flexibilidad y agilizan el desarrollo.

Por estos motivos Angular es la mejor solución en el caso de este software y en caso de quedarse corta en algunos aspectos, permite la integración de otras librerías. Además Angular incluye librerías y mecanismos para tests unitarios e integración continua, algo muy necesario en el desarrollo de un software.

Finalmente para los test unitarios y end-to-end (e2e) se van a utilizar herramientas que se integran perfectamente con Angular. Para los primeros se utiliza Karma, que viene incluido por defecto en el framework y permite comprobar el funcionamiento unitario de cada uno de los componentes. Para los tests e2e se utilizará Cypress, una herramienta gratuita muy potente que permite realizar tests al frontend como si de un usuario se tratara, haciendo uso de datos sobre las diferentes funcionalidades del sistema.






\section{Base de datos}


Elegir un sistema de gestión de base de datos es una de las decisiones más importantes a la hora de diseñar y desarrollar un software. Existe una gran variedad de tipos de bases de datos y hay que tener en cuenta una serie de cuestiones que serán determinantes a la hora de elegir un tipo u otro. En este proyecto se va a usar una base de datos NoSQL, concretamente MongoDB.

Las bases de datos de tipo SQL se basan en las relaciones entre los datos. Estos se introducen en registros y luego se organizan por tablas, columnas y tuplas, permitiendo relacionarlos de manera sencilla. El principal lenguaje de consultas es el \textit{Standard Query Language} (SQL), el cual esta compuesto por una serie de comandos de diferentes tipos, que se usan para unos cometidos u otros. Sus principales características son el esquema rígido que se define previo al uso que garantiza el esquema ACID.

Por las características del proyecto este modelo queda excluido, ya que el tipo de datos que se va a manejar no requiere de grandes relaciones entre ellos y además el esquema puede ser cambiante. Podría ocurrir que ciertos valores no se encontraran almacenados, bien porque no son necesarios o bien porque el usuario decide no insertarlos, por lo que sería mantener una estructura que no se está cumpliendo.

Por otro lado el tipo de consultas que se van a realizar no son extremadamente complejas. Los datos manejados no tienen relaciones entre sí y las consultas serían realmente básicas. Otro punto a favor en este aspecto para las bases de datos NoSQL es la velocidad a la hora de realizar las consultas.

En el caso de la integración con el software en las bases de datos SQL se utilizan los llamados ORM (Object Relation Mapper). Estos permiten realizar consultas a estas bases de datos de una forma más amigable en el lenguaje que se esté usando, lo que implica que se tenga que volver a redefinir el esquema para poder manejar estos datos. En cambio con NoSQL esta integración suele ser mas sencilla, al utilizarse directamente objetos como diccionarios.

Dentro de las bases de datos NoSQL existen diferentes tipos. En el caso de este proyecto el modelo que mas encaja es el documental, en la que una semiestructura flexible almacenada en forma de documentos es ideal. MongoDB es una gran elección, ya que los datos se almacenan en BSON (Binary JSON), lo que ofrece aún más flexibilidad a la hora de almacenar objetos. También permite crear índices en cualquier clave y el balanceo de carga en el caso de realizar grandes cantidades de consultas simultáneas.

Actualmente también están destacando las bases de datos en la nube o DBaaS (DataBase as a Service), las cuales estan optimizadas para operaciones en entornos virtualizados. La principal característica de estos servicios es que se suele pagar por el uso de almacenamiento y además conceptos como la escalabilidad o la alta disponibilidad estan asegurados. En el caso de MongoDB existe \textbf{Atlas}, que incluso ofrece planes gratuitos. Otro ejemplo de DBaaS sería \textbf{mLab}, muy similar al anterior pero con una configuración más sencilla. Para el desarrollo de este software este aspecto es muy interesante, ya que al tratarse de un microservicio el no estar atado a una base de datos local permite que se pueda desplegar tambien en la nube.

	
	\chapter{Solución propuesta}

En este capítulo se expone la solución que se propone al problema. En primer lugar se desarrollan todos los módulos que compondrían el backend y finalmente el frontend.




\section{Almacenamiento}

La principal idea en cuanto al almacenamiento es tener algún tipo de clase que sirva de conector de cualquier tipo de dato que se quiera almacenar con el almacén de datos y que a su vez permita agregar nuevos elementos de forma sencilla.

Desarrollar una estructura de herencia es el camino a seguir en este caso, el siguiente sería intentar lograr la mayor versatilidad posible.

Para facilitar el desarrollo de los diferentes módulos se propone desarrollar una base común que sirva como puente para realizar operaciones en las diferentes colecciones de la base de datos.

Esta podría llamarse \textit{Item} e implementaría las cuatro operaciones básicas necesarias para manejar cualquier tipo de dato: crear, modificar, obtener y eliminar. Una vez implementados esos métodos básicos el resto de módulos que se desarrollen sólo tienen que sobreescribir las operaciones necesarias para adecuarlas a cada uso concreto.

Por otro lado, para manejar el cliente de \textit{MongoDB} se propone crear una clase, llamada \textit{MongoEngine}, que permita realizar diferentes operaciones en las colecciones y bases de datos. Además, podría obtenerse también algún tipo de estadísticos de este servicio. Ambos módulos funcionarían conjuntamente para ofrecer un conector a la base de datos sencillo y capaz de adaptarse a cualquier tipo de uso.

Para la configuración de cada clase que pueda heredar de \textit{Item} se debería definir un nombre de la colección a usar por esa clase y además el esquema de la colección. Este esquema sería el conjunto de datos que se pueden almacenar en la colección.


\bigskip
En conclusión, se propone:
\begin{itemize}
	\item Clase \textit{Item}
	\item Clase \textit{MongoEngine}
\end{itemize}


\section{Autenticación}

El objetivo de este módulo es agregar una capa de seguridad al backend evitando que puedan acceder a él usuarios que no se encuentran registrados. Esta capa se aplicaría a todos los \textit{endpoints} que se quieran proteger de accesos indeseados.

Se propone por tanto un módulo que permita la autenticación de usuarios en el sistema. Este haría uso de \textit{JWT} (\textit{JSON Web Token}) para encriptar la información. Debido a que el backend es una \textit{API REST} el método de enviar este \textit{token} en cada una de las peticiones será la inclusión de este en cabeceras de las peticiones que se realicen. La cabecera a usar podría ser: \textit{x-access-token}.

En cada petición este \textit{token} deberá ser decodificado, se comprobará al usuario al que pertenece y finalmente se permitirá el acceso o no. Todos los \textit{endpoints} del backend estarían protegidos por esta autenticación, salvo:
\begin{itemize}
	\item \textit{GET /login}
	\item \textit{GET /api/heartbeat}
\end{itemize}


\bigskip
El servicio de autenticación a implementar debería implementar:
\begin{itemize}
	\item \textit{GET /login}
	\item \textit{GET /logout}
\end{itemize}



\section{Clientes}

Los clientes son aquella unidad que permite diferenciar un conjunto de datos de otro ya que cada uno de ellos cuenta con su propia base de datos. 

Debido a la \textit{API REST} que proporciona el backend se propone que para acceder a un cliente u otro se utlice el subdominio de la \textit{URL} a la que se le hacen las peticiones.

Por tanto, este módulo sería el encargado de diferenciar y manejar los diferentes clientes que podrían acceder al backend. La clase \textit{Customer} es la propuesta en este caso. Heredaría las funcionalidades de \textit{Item} y las complementaría con la gestión de estos clientes.

Para almacenar la información de estos clientes se propone también tener un ``cliente base'' el cual se encargaría de tener un registro de cada uno de los clientes disponibles en el sistema y de la base de datos que utiliza cada uno.



\section{Usuarios}

Éste módulo es el encargado de la gestión de los usuarios asociados a un cliente y sus funcionalidades serían las siguientes:
\begin{itemize}
	\item Crear usuarios
	\item Modificar usuarios
	\item Obtener información de los usuarios
	\item Eliminar usuarios
\end{itemize}


\bigskip
Para satisfacer los requisitos del software se propone lo siguiente:
\begin{itemize}
	\item Los datos a almacenar por usuario son: Identificador único, Tipo de usuario, Nombre, Apellidos, Email, Nombre de usuario, Contraseña.
	\item La contraseña se encriptaría con encriptado simétrico \textit{Fernet}.
	\item Los tipos de usuario aceptados serían 'admin' y 'regular'.
	\item El email será único.
\end{itemize}



\bigskip
Debido a que partimos del módulo \textit{Item} para desarrollar el de usuarios solo es necesario heredar de éste y hacer algunas modificaciones. En cuanto al borrado y obtención de usuarios no sería necesario hacer ningún tipo de modificación. Por otro lado, en las operaciones de inserción y modificación sólo habría que añadir el código necesario para el encriptado de la contraseña y para la comprobación del tipo de usuario y del email único.

\bigskip
Este módulo contaría con los siguientes \textit{endpoints}:
\begin{itemize}
	\item \textit{GET /user/:user} - Obtener la información de un usuario.
	\item \textit{POST /user} - Crear un usuario.
	\item \textit{PUT /user} - Modificar un usuario.
	\item \textit{DELETE /user} - Eliminar un usuario.
	\item \textit{POST /user/query} - Listar usuarios.
\end{itemize} 





\section{Despliegues}

Éste módulo sería el encargado de realizar los despliegues de los servicios mediante contenedores \textit{Docker}. Para llevar a cabo esto el módulo se conectaría a un servidor de \textit{Docker} y contendría los métodos necesarios para ejecutar contenedores, imágenes y operaciones en ambos.

En el caso de los despliegues no es necesario el almacenamiento de datos de ningún tipo por lo que tampoco sería necesario crear clases que hereden de \textit{Item}. En cambio, se propone la creación de una clase, llamada \textit{DockerEngine}, que permita conectarse al cliente de \textit{Docker} e implemente los métodos necesarios para hacer las operaciones deseadas.

Estas serían:
\begin{itemize}
	\item Ejecutar operaciones en todos los contenedores.
	\item Ejecutar operaciones en un contenedor en concreto.
	\item Ejecutar operaciones en todas las imágenes.
	\item Ejecutar operaciones en una imagen en concreto.
\end{itemize}


\bigskip
Las anteriores operaciones corresponderían con los siguientes \textit{endpoints}:
\begin{itemize}
	\item \textit{POST /deploy/container}
	\item \textit{POST /deploy/container/single}
	\item \textit{POST /deploy/image}
	\item \textit{POST /deploy/image/single}
\end{itemize}


\bigskip
Además, del mismo modo que se propone en el módulo de almacenamiento, podrían obtenerse una serie de datos estadísticos de este servicio.





\section{Aprovisionamiento}
Sería el encargado de aprovisionar sistemas mediante el uso de Ansible y se centraría exclusivamente en la ejecución de \textit{playbooks}. Por el funcionamiento de \textit{Ansible} debería establecer una conexión \textit{SSH} con los hosts indicados y ejecutaría las órdenes que se encuentran en el playbook.

En el caso de este módulo son necesarias dos clases extra, una para almacenar los \textit{playbooks} y otra para almacenar los grupos de hosts donde se van a ejecutar esos \textit{playbooks}. Las clases serían:
\begin{itemize}
	\item \textit{Hosts}
	\item \textit{Playbooks}
\end{itemize}


\bigskip
Los \textit{endpoints} que se proponen para manejar ambas clases son:
\begin{itemize}
	\item \textit{GET /provision/hosts/:name} - Obtener la información de un grupo de \textit{hosts}.
	\item \textit{POST /provision/hosts} - Crear un grupo de \textit{hosts}.
	\item \textit{PUT /provision/hosts} - Modificar un grupo de \textit{hosts}.
	\item \textit{DELETE /provision/hosts} - Eliminar un grupo de \textit{hosts}.
	\item \textit{POST /provision/hosts/query} - Listar grupos de \textit{hosts}.
	\item \textit{GET /provision/playbook/:name} - Obtener la información de un \textit{Playbook}.
	\item \textit{POST /provision/playbook} - Crear un \textit{Playbook}.
	\item \textit{PUT /provision/playbook} - Modificar un \textit{Playbook}.
	\item \textit{DELETE /provision/playbook} - Eliminar un \textit{Playbook}.
	\item \textit{POST /provision/playbook/query} - Listar \textit{playbooks}.
\end{itemize}

\bigskip
Siguiendo los requisitos del software, las restricciones son:
\begin{itemize}
	\item Se almacenará para cada \textit{Playbook} un identificador único, un nombre y el \textit{Playbook} en sí.
	\item Se almacenará para cada grupo de \textit{playbooks} un identificador único, un nombre y el conjunto de direcciones IP asociadas.
\end{itemize}

\bigskip
Por otro lado, para ejecutar los playbooks se propone la creación de una clase \textit{AnsibleEngine}, que sería la encargada de implementar aquellos métodos necesarios para ejecutarlos. También se propone el siguiente \textit{endpoint}:
\begin{itemize}
	\item \textit{POST /provision}
\end{itemize}




\section{Máquinas}

Módulo encargado del almacenamiento y gestión de máquinas y dispositivos. Tendría estructura similar a las clases \textit{Host} o \textit{Playbook}, ya que heredaría las funcionalidades que ofrece la clase base \textit{Item}.

\bigskip
Los \textit{endpoints} propuestos para este módulo son:
\begin{itemize}
	\item \textit{GET /machine/:user} - Obtener la información de una máquina.
	\item \textit{POST /machine} - Crear una máquina.
	\item \textit{PUT /machine} - Modificar una máquina.
	\item \textit{DELETE /machine} - Eliminar una máquina.
	\item \textit{POST /machine/query} - Listar máquinas.
\end{itemize}



\bigskip
Requisitos del software:
\begin{itemize}
	\item El sistema almacenará para cada máquina un identificador único, un nombre, una descripción, un tipo de máquina, dirección IPv4 e IPv6, dirección MAC, máscara de red, dirección broadcast y dirección de red.
\end{itemize}





\section{Estado del backend y \textit{heartbeat}}


Para comprobar el estado del backend se propone la creación de un servicio que devuelva información asociada al modulo de despliegues y al de almacenamiento. Para ello se propone agregar métodos a las clases \textit{MongoEngine} y \textit{DockerEngine} que devuelvan esta información asociada.

\bigskip
El \textit{endpoint} sería el siguiente:
\begin{itemize}
	\item \textit{GET /status}
\end{itemize}

\bigskip
Debido a que este \textit{endpoint} devolvería información relevante, éste deberia estar también protegido por la autenticación comentada en secciones anteriores.

\bigskip
Anexo a este estado se propone el siguiente \textit{endpoint}:
\begin{itemize}
	\item \textit{GET /api/heartbeat}
\end{itemize}


\bigskip
En este caso solo devolvería si los diferentes módulos del backend se encuentran funcionando correctamente o no, y no sería necesario que estuviera autenticado. Este \textit{endpoint} podría ser usado por \textit{Docker} en el caso de que el backend se ejecute en un contenedor de este tipo.



\section{Variables de entorno}


Para el funcionamiento del backend y el frontend sería necesaria la definición de variables de entorno que permitan configurar ciertos aspectos de estos. Serían:
\begin{itemize}
	\item \textit{Hostname} y puerto de \textit{MongoDB}.
	\item Nombre de la base de datos a utilizar.
	\item Claves de encriptado para las contraseñas y los token de autenticación.
	\item \textit{Hostname} de \textit{Docker}.
	\item URL y puerto del backend.
\end{itemize}





\section{\textit{CLI}}

Módulo propuesto para poder realizar ciertas operaciones desde la terminal, sin necesidad de ejecutar el backend. Podría ser usado en la primera instalación de este y/o para crear unos primeros usuarios o clientes.

\bigskip
Funcionalidad propuesta:
\begin{itemize}
	\item Crear clientes.
	\item Activar o desactivar clientes.
	\item Agregar usuarios a un cliente.
\end{itemize}




\section{Frontend}

El desarrollo de un backend que ofrezca una \textit{API REST} permite que se pueda desarrollar cualquier tipo de frontend, ya sea web, una aplicación móvil o incluso acceso mediante línea de comandos. En este caso, para satisfacer los requisitos del software, se propone crear un frontend que permita realizar todas las operaciones anteriormente mencionadas.

\bigskip
Este podría tener las siguientes páginas:
\begin{itemize}
	\item \textit{/}: Donde mostrar el estado general del sistema.
	\item \textit{/admin}: Administración de usuarios.
	\item \textit{/deploy}: Administración de los despliegues, contenedores e imágenes.
	\item \textit{/provision}: Administración del aprovisionamiento, grupos de hosts y playbooks.
	\item \textit{/machines}: Administración de las máquinas.
\end{itemize}

\bigskip
Por otro lado, atendiendo a los requisitos del software, se propone la creación de un componente para generar tablas de forma dinámica que encapsule todas las funcionalidades requeridas y que además permita tener una sincronía en la forma de mostrar los datos al usuario. Además debe incluir autenticación de los usuarios.

Para la comunicación con la \textit{API} se propone la creación de diferentes servicios centrados en cada uno de los módulos del backend. De esta manera los servicios pueden inyectarse en los componentes y la comunicación es directa. Estos serían:
\begin{itemize}
	\item Autenticación
	\item Clientes
	\item Usuarios
	\item Hosts
	\item Playbooks
	\item Máquinas
	\item Aprovisionamiento
	\item Despliegues
	\item Estado
\end{itemize}


	
	\chapter{Implementación}



\section{Almacenamiento}

\subsection{MongoEngine}
\label{sec:mongoengine}

Esta clase es la que nos permite conectarnos al servidor de \textit{MongoDB} y realizar todas aquellas operaciones que deseemos. Se ha desarrollado como un singleton para que solo haya una instancia activa al mismo tiempo.

Cada instancia de \textit{MongoClient} que se crea tiene una \textit{pool} de conexiones, que abre y cierra sockets bajo demanda para manejar todas las operaciones que se realicen de forma simultánea. Por este motivo es contraproducente crear una instancia de \textit{MongoClient} cada vez que se quiera realizar una operación. Por defecto el tamaño de esta \textit{pool} es de 100, pero podría incrementarse si fuera necesario.

\bigskip
Los métodos implementados son los siguientes:
\begin{itemize}
	\item Creación del cliente.
	\item Borrado de bases de datos y de colecciones. Usados principalmente en los tests unitarios.
	\item Asignación de base de datos y colección actual. Usados para seleccionar la base de datos necesaria para cada cliente, y la colección donde realizar operaciones.
	\item Datos estadísticos. Se extraen del cliente de \textit{MongoDB} y se devuelve en un diccionario con la siguiente forma:
\end{itemize}

\pagebreak
\begin{lstlisting}
{
	"is_up": bool,
	"data_usage": list,
	"info": dict || str
}
\end{lstlisting}

\begin{itemize}
	\item \textit{is\_up}: Indica el estado del servicio, true si se encuentra funcionando correctamente, false en caso contrario.
	\item \textit{data}: Información de uso de datos de las bases de datos almacenadas.
	\item \textit{info}: Información adicional del cliente.
\end{itemize}



\subsection{Item}

Clase base de la que heredan todas las clases que necesitan algún tipo de almacenamiento de datos. Sus datos miembro son:
\begin{itemize}
	\item \textit{table\_name}: Nombre de la tabla (o colección) donde se van a almacenar los datos.
	\item \textit{table\_schema}: Esquema de la tabla equivalente a la proyección de \textit{MongoDB}. Es un diccionario compuesto por claves (nombres de los datos que va a almacenar la tabla) y por un valor 1 ó 0. Todas las claves que aparezcan en este diccionario serán claves válidas para almacenar en la tabla. Los valores indican lo siguiente:
	\begin{itemize}
		\item \textit{1}: El dato se devuelve al hacer una consulta.
		\item \textit{0}: El dato no se devuelve al hacer una consulta.
	\end{itemize}
\end{itemize}

\bigskip
Un ejemplo de uso sería:

\begin{lstlisting}
table_schema = {
	'domain': 1,
	'db_name': 1
}
\end{lstlisting}

\bigskip
Al realizar consultas se puede sobrescribir este esquema, para obtener únicamente los datos deseados.
\begin{itemize}
	\item \textit{data}: Diccionario donde se almacenan los datos de los objetos que se creen de este tipo o que se obtengan al hacer una consulta.
\end{itemize}


\bigskip
Para el diseño y desarrollo de esta clase se ha intentado abstraer y simplificar al máximo las funcionalidades de esta, para que se pueda adaptar a cualquier tipo de uso. No se han hecho uso de todas las funciones que ofrece \textit{PyMongo} en cuanto a manejo de datos en colecciones. Aún así los métodos son los mínimos para que se pueda cualquier tipo de operación básica. Los métodos implementados en la clase \textit{Item} son los siguientes:
\begin{itemize}
	\item \textit{cursor}: Hace uso del nombre de la tabla para obtener el cursor a ella y poder realizar todas las operaciones necesarias.
	\item \textit{find(criteria, projection)}: Devuelve todos los elementos (sólo los parámetros indicados en la proyección) que cumplan os criterios de búsqueda.
	\item \textit{insert(data)}: Inserta un elemento o lista de elementos en la colección. También agrega dos claves adicionales a éste que son:
	\item \textit{enabled}: Por defecto a true. Indica si el elemento está activo o no.
	\item \textit{deleted}: Por defecto a false. Indica si el elemento ha sido borrado o no.
	\item \textit{update(criteria, data)}: Actualiza todos los elementos que cumplan con el criterio. \textit{Data} es un diccionario con las claves y valores a actualizar.
	\item \textit{remove(criteria, force)}: Elimina los elementos que cumplan con el criterio. Por defecto el parámetro \textit{force} tiene valor \textit{true}, lo que elimina completamente los elementos. En el caso de asignarle un valor false en lugar de eliminar los elementos los modifica, cambiando sus propiedades \textit{enabled} a \textit{false} y \textit{deleted} a \textit{true}.
\end{itemize}



\bigskip
Con esta implementación cualquier clase que se quiera que tenga la capacidad de almacenar datos solo tendrá que heredar de esta clase. Podrá sobrescribir aquellos métodos a los que quiera agregar más funcionalidad y podrá también implementar nuevos métodos, ya que tiene acceso al cursor de \textit{MongoClient} para realizar cualquier tipo de operación permitida.




\section{Servicios}
\label{sec:servicios}


En cuanto a los servicios que se han desarrollado a partir de clases que heredan de \textit{Item}, la estructura general de todos es la siguiente:

En los siguientes \textit{endpoints} suponemos que el servicio \textit{service} permite hacer operaciones con objetos de tipo \textit{Service}:
\begin{itemize}
	\item \textit{GET /service/:name}: Devuelve los datos asociados a un objeto de tipo \textit{Service}.
	\item \textit{POST /service}: Crea un objeto de tipo \textit{Service}.
	\item \textit{PUT /service}: Modifica un objeto de tipo \textit{Service}.
	\item \textit{DELETE /service}: Elimina un objeto de tipo \textit{Service}.
	\item \textit{POST /service/query}: Permite hacer consultas más elaboradas haciendo uso del criterio de búsqueda y de la proyección de \textit{MongoDB}. Los objetos obtenidos se devuelven en forma de diccionario (en el caso de un sólo resultado) o de lista de diccionarios (más de un resultado).
\end{itemize}


\bigskip
Para simplificar el código en los servicios he creado algunas funciones auxiliares:
\begin{itemize}
	\item \textit{response\_by\_success}: Devuelve un mensaje predeterminado y un código en función del resultado de la operación que se haya procesado.
	\item \textit{response\_with\_message}: Devuelve un mensaje y código personalizados.
	\item \textit{validate\_or\_abort}: Valida los datos de entrada del endpoint en función del esquema que se quiera validar.
	\item \textit{parse\_data}: Devuelve los datos que se le pasan con la forma del esquema que se quiera. Tiene en cuenta si es un solo dato o un conjunto.
\end{itemize}


\bigskip
Por otro lado, se han creado diferentes esquemas con \textit{Marshmallow} para validar todos los datos que se manejan en los servicios. De este modo se tiene control absoluto del tipo de dato que se esté manejando en cada momento. En el momento de recibir una petición los datos se cotejan con el esquema que se use en el endpoint y se asegura que los datos son los esperados, en caso contrario se rechaza la petición, para ello se hace uso de la función anterior \textit{validate\_or\_abort}. En cuanto a la salida de datos se usa \textit{parse\_data} para asegurar que los datos devueltos tenga la estructura esperada.


\section{Autenticación}

Utilizado para sólo permitir el uso de la \textit{API} a los usuarios registrados. Esta autenticación se hace mediante \textit{JWT} (\textit{JSON Web Token}).

Para la creación de esta clase se ha partido de la clase \textit{Item}, definiendo una nueva clase \textit{Login} con los siguientes datos miembro:
\begin{itemize}
	\item \textit{table\_name}: login
	\item \textit{table\_schema}: (Por defecto todos los valores a 1).
	\begin{itemize}
		\item \textit{token}: \textit{JWT} del usuario que tenga acceso actualmente al backend.
		\item \textit{username}: Nombre de usuario.
		\item \textit{exp}: Fecha y hora a la que expira el acceso.
		\item \textit{login\_time}: Fecha y hora a la que el usuario realizó el acceso.
		\item \textit{public\_id}: \textit{UUID} que identifica al usuario.
	\end{itemize}
\end{itemize}



\bigskip
Para controlar los accesos que puedan quedar obsoletos o en los que no se haya realizado un logout correcto, cada vez que se instancia la clase se comprueba si hay tokens con estas características y se eliminan, evitando así una acumulación innecesaria de tokens sin usar.

\bigskip
Los métodos que se han desarrollado han sido los siguientes:
\begin{itemize}
	\item \textit{login(auth)}: En este método se realiza todo el proceso del login. Se parte de los datos de autenticación, compuestos por un usuario y una contraseña y se comprueba si tal usuario existe. Se verifica que la contraseña sea la correcta y en caso correcto se procede a la creación del token. Tanto como si el usuario no estaba logueado previamente como si ya lo estaba, se generan todos los datos asociados nuevamente, y se insertan o actualizan. En el token se codifica el \textit{public\_id} y la fecha y hora de expiración, \textit{exp}. Si el proceso ha sido correcto se devolverá el token creado.
	\item \textit{logout(username)}: Desloguea al usuario denotado por \textit{username}. Verifica que existe el usuario y borra cualquier tipo de información asociada de eśte en la colección actual.
	\item \textit{token\_access(token)}: Decodifica el token y devuelve al usuario logueado que tenga tal \textit{public\_id}.
	\item \textit{get\_username(token)}: Devuelve el nombre del usuario asociado al token.
\end{itemize}


Una vez desarrollada la clase que permite accesos de usuarios se ha desarrollado el servicio. Todas las rutas del backend, salvo \textit{GET /login} y \textit{GET /api/heartbeat} están protegidas con esta autenticación. Para ello se ha creado un decorador que comprueba el token de acceso cada vez que se quiere acceder a un endpoint. Es el siguiente:
\begin{itemize}
	\item \textit{token\_required}: Obtiene el token de la cabecera \textit{x-access-token}, comprueba si es válido y permite el acceso o no al endpoint. En caso de no ser válido se devuelve un mensaje de error y un código de error 401.
\end{itemize}


\bigskip
El servicio de login cuenta con dos \textit{endpoints}, los cuales son:
\begin{itemize}
	\item \textit{GET /login}: Loguea al usuario, para ello toma los datos de acceso de la cabecera \textit{Basic auth} y devuelve o no el token asociado al usuario.
	\item \textit{GET /logout}: Desloguea al usuario que previamente debe estar logueado.
\end{itemize}




\section{Clientes}

Clase que se encarga del manejo de los \textit{customers} o clientes del backend. Todas las operaciones relacionadas con clientes así como los datos asociados a ellos se almacenan en la base de datos que se define en la variable de entorno \textit{BASE\_DATABASE}.

\bigskip
Los datos miembro de esta clase son:
\begin{itemize}
	\item \textit{table\_name}: customers
	\item \textit{table\_schema}: (Por defecto todos los valores a 1).
	\begin{itemize}
		\item \textit{domain}: Subdominio al que hace referencia este cliente.
		\item \textit{db\_name}: Nombre de la base de datos donde se almacenarán todos sus datos.
	\end{itemize}
\end{itemize}


\bigskip
Los métodos que implementa esta clase son:
\begin{itemize}
	\item \textit{is\_customer(customer)}: Comprueba si el customer existe. En caso de existir se devuelve si está activo o no.
	\item \textit{set\_customer(customer)}: Asigna el customer al que se le van a hacer consultas de base de datos. Esto es: se consulta el cliente en \textit{BASE\_DATABASE} y se obtiene su \textit{db\_name}, a continuación se asigna este nombre de colección como la base de datos a utilizar, haciendo uso del método \textit{set\_collection\_name} de \textit{MongoEngine}.
	\item \textit{insert}: Se ha sobreescrito este método de \textit{Item} para realizar comprobaciones previa inserción de nuevos clientes.
	\item \textit{find}: Se ha sobreescrito para asegurar que las operaciones se hacen sobre \textit{BASE\_DATABASE}.
	\item \textit{update}: Se ha sobreescrito para asegurar que las operaciones se hacen sobre \textit{BASE\_DATABASE}.
	\item \textit{remove}: Se ha sobreescrito para asegurar que las operaciones se hacen sobre \textit{BASE\_DATABASE}.
\end{itemize}


\bigskip
Los \textit{endpoints} desarrollados tienen la forma que se indica \hyperref[sec:servicios]{aquí}, pero no se han implementado todos ellos, sólo los siguientes:
\begin{itemize}
	\item \textit{POST /customer}
	\item \textit{PUT /customer}
	\item \textit{DELETE /customer}
	\item \textit{POST /customer/query}
\end{itemize}


\bigskip
Por otro lado, para controlar el cliente que se debe utilizar en cada petición se ha creado una función para este cometido. Obtiene el subdominio del host de la petición y comprueba si se trata de un cliente válido; en caso afirmativo se asigna como cliente para esa petición y en caso negativo se aborta la petición con un código 404.




\section{Usuarios}


Esta es la clase encargada del manejo de los usuarios y hereda de \textit{Item}. Trabaja junto con la clase \textit{Login} para permitir el acceso a la \textit{API}. Los datos miembro de esta clase son:
\begin{itemize}
	\item \textit{table\_name}: users
	\item \textit{table\_schema}: (Por defecto todos los valores a 1).
	\begin{itemize}
		\item \textit{type}: Tipo del usuario, puede ser \textit{admin} o \textit{regular}.
		\item \textit{first\_name}: Nombre del usuario.
		\item \textit{last\_nam}e: Apellido/s del usuario.
		\item \textit{username}: Nickname del usuario.
		\item \textit{email}: Email del usuario.
		\item \textit{password}: Contraseña del usuario.
		\item \textit{public\_id}: \textit{UUID} del usuario.
	\end{itemize}
\end{itemize}


\bigskip
Se han sobreescrito los métodos de inserción y actualización de datos para tener en cuenta las restricciones de tipo de usuario, para la generación del \textit{UUID} y para el cifrado de la contraseña. Este cifrado se hace con \textit{Fernet}, el cual es de tipo simétrico.

En cuanto al servicio, este está estructurado de la misma forma que se especifica \hyperref[sec:servicios]{aquí}, siendo los \textit{endpoints}:
\begin{itemize}
	\item \textit{GET /user/:username}
	\item \textit{POST /user}
	\item \textit{PUT /user}
	\item \textit{DELETE /user}
	\item \textit{POST /user/query}
\end{itemize}




\section{Despliegues}

Este módulo es el encargado de realizar los despliegues de los servicios mediante contenedores \textit{Docker}. Para llevar a cabo esto se conecta a un servidor de \textit{Docker} y contiene los métodos necesarios para ejecutar contenedores, imágenes y operaciones en ambos.

Actualmente no permite realizar algunas funciones, como son el manejo de redes, nodos, o volúmenes. Una futura mejora o ampliación del módulo podría incluir estas u otras nuevas funcionalidades. Por el momento no eran necesarias y se han priorizado los contenedores y las imágenes. Por otro lado también permite obtener información del estado del cliente, del uso de almacenamiento e información general.

\bigskip
Contextualización:
\begin{itemize}
	\item En este módulo se entiende por \textbf{cliente} al conector que se crea en el sistema para comunicarnos con el \textit{daemon} de \textit{Docker}.
	\item Un \textbf{objeto} puede ser una imagen o un contenedor.
	\item Por \textbf{operación} se entiende toda aquella tarea que se puede ejecutar en un contenedor o en una imagen.
\end{itemize}

\bigskip
Las operaciones constan de nombre, datos y, en ocasiones, de un objeto concreto:
\begin{itemize}
	\item \textit{operation}: Nombre de la operación.
	\item \textit{data}: Datos necesarios para ejecutar la operación.
	\item \textit{object}: Objeto al que se dirige la operación.
\end{itemize}


\bigskip
Para manejar esto se ha creado la clase \textit{DockerEngine}, la cual se conecta al \textit{daemon} de \textit{Docker} e implementa los métodos necesarios para realizar las funciones anterior mencionadas. Concretamente estas son:
\begin{itemize}
	\item \textit{run\_container\_operation(operation, data)}: Ejecuta una operación en todos los contenedor.
	\item \textit{run\_image\_operation(operation, data)}: Ejecuta una operación en todas las imágenes.
	\item \textit{run\_operation\_in\_object(object, operation, data)}: Ejecuta una operación en todos los contenedor.
	\item \textit{get\_container\_by\_id(name)}: Ejecuta una operación en todos los contenedor.
	\item \textit{get\_image\_by\_name(container\_id)}: Ejecuta una operación en todos los contenedores.
\end{itemize}


\bigskip
Los \textit{endpoints} desarrollados son los siguientes:
\begin{itemize}
	\item \textit{POST /deploy/container}: Operaciones en todos los contenedores.
	\item \textit{POST /deploy/container/single}: Operaciones en un único contenedor.
	\item \textit{POST /deploy/image}: Operaciones en todas las imágenes.
	\item \textit{POST /deploy/image/single}: Operaciones en una única imagen.
\end{itemize}


\bigskip
La información sobre el estado de este módulo se devuelve con la misma estructura que se utiliza en \hyperref[sec:mongoengine]{\textit{MongoEngine}}:
\begin{lstlisting}
{
	"is_up": boolean,
	"data_usage": list,
	"info": dict || string
}
\end{lstlisting}



\bigskip
Puede ocurrir que el backend se esté ejecutando en un contenedor \textit{Docker}, por lo que intentar conectarse al \textit{daemon} no sería posible en ese caso. Cuando se ejecuta el backend éste comprueba internamente en qué entorno se esta ejecutando y la variable que almacena la localización del \textit{daemon} o servidor. Estas comprobaciones determinarán si éste módulo se encontrará activo o no.




\section{Aprovisionamiento}


Es el encargado de aprovisionar sistemas mediante el uso de Ansible y se centra exclusivamente en la ejecución de \textit{playbooks} y en la gestión y almacenamiento de grupos de \textit{hosts} y \textit{playbooks}.
\begin{itemize}
	\item Un \textbf{grupo de \textit{hosts}} es aquella máquina o grupo de máquinas que se quiere aprovisionar.
	\item Un \textbf{playbook} es el conjunto de ordenes, comandos y tareas que se quiere que se ejecuten en un grupo de hosts.
\end{itemize}

Tanto los \textit{hosts} como los \textit{playbooks} se pueden almacenar en base de datos si se desea y por tanto se han desarrollado las clases \textit{Hosts} y \textit{Playbook}.



\subsection{Hosts}

Clase que hereda de \textit{Item} cuyos datos miembro son:
\begin{itemize}
	\item \textit{table\_name}: hosts
	\item \textit{table\_schema}: (Por defecto todos los valores a 1).
	\begin{itemize}
		\item \textit{name}: Nombre del grupo de \textit{hosts}.
		\item \textit{ips}: Lista de direcciones IP que componen el grupo de \textit{hosts}.
	\end{itemize}
\end{itemize}
	
\bigskip
Comparte los mismos \textit{endpoints} que se indican \hyperref[sec:servicios]{aquí}, siendo estos:
\begin{itemize}
	\item \textit{GET /provision/hosts/:name}
	\item \textit{POST /provision/hosts}
	\item \textit{PUT /provision/hosts}
	\item \textit{DELETE /provision/hosts}
	\item \textit{POST /provision/hosts/query}
\end{itemize}


\subsection{Playbook}

Clase que hereda de \textit{Item} cuyos datos miembro son:
\begin{itemize}
	\item \textit{table\_name}: \textit{playbooks}
	\item \textit{table\_schema}: (Por defecto todos los valores a 1).
	\begin{itemize}
		\item \textit{name}: Nombre del \textit{playbook}.
		\item \textit{playbook}: Contenido del \textit{playbook} codificado como \textit{JSON}.
	\end{itemize}
\end{itemize}
	
\bigskip
Comparte los mismos \textit{endpoints} que se indican \hyperref[sec:servicios]{aquí}, siendo estos:
\begin{itemize}
	\item \textit{GET /provision/playbooks/:name}
	\item \textit{POST /provision/playbooks}
	\item \textit{PUT /provision/playbooks}
	\item \textit{DELETE /provision/playbooks}
	\item \textit{POST /provision/playbooks/query}
\end{itemize}

\bigskip
Para la ejecución de los playbooks se ha desarrollado la clase \textit{AnsibleEngine}, la cual implementa un método para este cometido, es:
\begin{itemize}
	\item \textit{run\_playbook(hosts, playbook, passwords)}
\end{itemize}

\bigskip
Este método toma como entrada el grupo de \textit{hosts} a los que se le va a ejecutar el \textit{playbook}, el \textit{playbook} en cuestión y un diccionario con las contraseñas para acceder a las máquinas. La librería de \textit{Ansible} usada requiere que los \textit{hosts} se pasen como un archivo de texto plano, por lo que se ha desarrollado una función auxiliar que crea un fichero cuando se ejecuta un \textit{playbook}. El directorio donde se guardan estos archivos puede configurarse mediante una variable de entorno.



\bigskip
El endpoint creado es el siguiente:
\begin{itemize}
	\item \textit{POST /provision}
\end{itemize}




\section{Máquinas}

Para el almacenamiento y gestión de máquinas se ha creado la clase \textit{Machine}, la cual también hereda de \textit{Item}. Sus datos miembro son:
\begin{itemize}
	\item \textit{table\_name}: machines
	\item \textit{table\_schema}: (Por defecto todos los valores a 1).
	\begin{itemize}
		\item \textit{name}: Nombre de la máquina.
		\item \textit{description}: Descripción breve de la máquina.
		\item \textit{type}: Tipo de máquina, puede ser \textit{local} o \textit{remote}.
		\item \textit{ipv4}: Dirección IPv4 de la máquina.
		\item \textit{ipv6}: Dirección IPv6 de la máquina.
		\item \textit{mac}: MAC del adaptador de red que conecta la máquina a la red.
		\item \textit{broadcast}: Dirección broadcast de la red a la que está conectada la máquina.
		\item \textit{netmask}: Máscara de red.
		\item \textit{network}: Red a la que está conectada la máquina.
	\end{itemize}
\end{itemize}


\bigskip
En el caso de las máquinas todas las direcciones IP que se manejan deben ser validadas, por lo que se ha creado un método auxiliar para realizar esta función. Además se han sobrescrito los métodos de inserción y actualización para realizar esta validación.

\bigskip
El servicio tiene la misma estructura que la explicada \hyperref[sec:servicios]{aquí} y sus \textit{endpoints} son:
\begin{itemize}
	\item \textit{GET /machine/:name}
	\item \textit{POST /machine}
	\item \textit{PUT /machine}
	\item \textit{DELETE /machine}
	\item \textit{POST /machine/query}
\end{itemize}



\section{Estado del backend y \textit{heartbeat}}


Para consultar el estado del backend se han creado dos \textit{endpoints}. El primero ofrece una información más detallada de los dos servicios más importantes y el segundo ofrece sólo el estado general del backend. Son:
\begin{itemize}
	\item \textit{GET /status}: \textit{Endpoint} autenticado que devuelve un diccionario con los estados de \textit{MongoDB} y \textit{Docker}, con la forma indicada anteriormente. Los usuarios administradores obtienen más información en el caso de \textit{MongoDB}. La estructura devuelta es:

\begin{lstlisting}
{
	'mongo': {
		'is_up': bool,
		'data_usage': list,
		'info': dict || str
	},
	'docker': {
		'is_up': bool,
		'data_usage': list,
		'info': dict || str
	}
}
\end{lstlisting}

\bigskip	
En el caso que \textit{Docker} no se encuentre funcionando correctamente o se encuentre desactivado el estado devuelto es:
	
\begin{lstlisting}
{
	'status': bool,
	'disabled': bool,
	'msg': str
}
\end{lstlisting}
	
	\item \textit{GET /api/heartbeat}: \textit{Endpoint} no autenticado que devuelve el estado simplificado. Este endpoint puede ser usado para comprobar que el backend se encuentra activo de forma manual o por ejemplo por la funcionalidad \textit{Heartbeat} que incorpora Docker para conocer el estado de salud de los contenedores.
	
\begin{lstlisting}
{
	'ok': bool
}
\end{lstlisting}
\end{itemize}



\section{\textit{CLI} e inicialización de la base de datos}


Se ha desarrollado un \textit{CLI} interactivo que permite realizar operaciones básicas con los clientes y usuarios mediante una terminal de comandos. Para ello hace uso de las clases y métodos anterior explicados. Estas operaciones son:
\begin{itemize}
	\item Crear, activar y desactivar clientes.
	\item Agregar usuarios a un cliente.
\end{itemize}


Además se ha creado un \textit{script} que permite inicializar la base de datos, insertando un usuario administrador por defecto.





\section{Frontend}

El frontend implementado es una de las infinitas propuestas que satisfacen los requisitos del software deseados. En las siguientes secciones se detallan los módulos que se han desarrollado.



\subsection{Servicios \textit{HTTP}}


Para la comunicación frontend-backend se han creado diferentes servicios, que de forma asíncrona realizan las peticiones \textit{HTTP} a la \textit{API}. Todos ellos hacen uso de \textit{HttpClient}, que es el múdulo de \textit{Angular} para realizar este tipo de peticiones. La estructura en general es muy similar, habiéndose creado métodos concretos para cada tipo de operación o endpoint. Cada uno de estos métodos tiene la forma:

\begin{lstlisting}
metodo(params: any): Observable<any> {
	return this.httpClient.put(url, {
		data
	}, {
		headers: access_token
	}).pipe(
	map(data => {
		return {
			ok: true,
			data
		};
	}),
	catchError(error => {
		return of({
			ok: false,
			error
		});
	})
	);
}
\end{lstlisting}

\bigskip
Se puede observar que:
\begin{itemize}
	\item Se hace una petición \textit{HTTP} (en el caso del ejemplo se usa \textit{PUT}) a través del módulo \textit{HttpClient} con diferentes argumentos. El primero es la \textit{URL} a la que se hace esa petición, el segundo los datos que se quieren pasar en el \textit{bod}y de esta y el tercero son las cabeceras, que en el caso de la \textit{API} desarrollada es necesaria la cabecera \textit{x-access-token} con el token de acceso.
	\item El observable que se devuelve es asíncrono, lo que quiere decir que se lanza la petición y que de forma asíncrona se completará la misma y se procesarán los datos.
	\item Se hace un \textit{pipe}. Esto permite hacer un primer procesado de los datos y en este caso se agrega una clave que indica que se han recibido los datos.
	\item Se capturan los posibles errores. En caso de que la petición no se complete o surja algún tipo de error se obtiene el error y se devuelve.
\end{itemize}


\bigskip
Los servicios desarrollados son los siguientes:

\subsubsection{URL}

Este servicio no es un servicio como tal, ya que no realiza peticiones \textit{HTTP}, pero sí se encarga del procesado  de la \textit{URL} a la que se hacen las peticiones. Es usado por el resto de servicios para obtener la \textit{URL} a la que tienen que hacer dichas peticiones. El procesado consiste en la creación de la \textit{URL} a partir del protocolo de acceso a la \textit{API} (\textit{HTTP} o \textit{HTTPS}), del cliente y de la \textit{URL} donde se encuentra la \textit{API}.

De este modo el resto de servicios solo tienen que hacer una llamada a este servicio para obtener la \textit{URL} actual.


\subsubsection{Autenticación}

La autenticación es el módulo más importante del frontend, ya que se encarga de toda la gestión del token y del acceso de los usuarios a las diferentes rutas de la aplicación. Este servicio se compone de dos métodos principales, login y logout, y de algunos secundarios. El funcionamiento de estos es:
\begin{itemize}
	\item \textit{login(auth)}: A partir de los datos de autenticación del usuario realiza el login y se almacena en las cookies del navegador el token de acceso. Se comprueba si existe el token en las cookies: en caso afirmativo se comprueba si es válido y autoriza o no el acceso; en caso contrario se hace una petición al endpoint \textit{GET /login} con los datos de acceso para obtener un nuevo token. Finalmente se almacena el token en las cookies. Debido a la asincronía de las peticiones este método también devuelve un observable.
	\item \textit{logout()}: Hace una petición de logout a la \textit{API} y cuando obtiene la respuesta borra el token de las cookies del navegador.
\end{itemize}



\subsubsection{Resto de servicios}

Los demás servicios tienen la estructura ya explicada al principio de esta sección, con métodos para hacer peticiones a los diferentes \textit{endpoints} de la \textit{API} y manejo de los posibles errores. Cuando ha sido preciso se han también métodos auxiliares. Estos servicios son:
\begin{itemize}
	\item Clientes
	\item Usuarios
	\item Hosts
	\item Playbooks
	\item Máquinas
	\item Aprovisionamiento
	\item Despliegues
	\item Estado del backend
\end{itemize}


\subsection{Guard}

Para proteger las rutas de usuarios no identificados se ha creado un \textit{Guard}. Este realiza una serie de comprobaciones acordes a nuestras necesidades antes de permitir o no el acceso a una página. En el caso de este frontend la única ruta que se quiere accesible por cualquier usuario es el login, por lo que el resto se han protegido.

Se ha hecho uso del método \textit{CanActivate} que proporciona \textit{Angular} para este cometido. Además, se han tenido en cuenta diferentes aspectos a la hora de diseñar y desarrollar esta funcionalidad, ya que por ejemplo no se debería poder acceder a los despliegues con \textit{Docker} si este esta desactivado en el backend.

Hace uso del \textit{router} que también proporciona Angular del servicio de autenticación, y su funcionamiento es el siguiente:
\begin{itemize}
	\item Si se quiere ir al \textit{login} se permite el acceso.
	\item En caso contrario se comprueba si el usuario está \textit{logeado}.
	\begin{itemize}
		\item Si lo está:
			\begin{itemize}
				\item Si se quiere acceder a la página de administración de usuarios se comprueba el usuario que está intentando acceder y se le permite o no el acceso.
				\item Si se quiere acceder a la página de los despliegues se comprueba si \textit{Docker} está activo en el backend.
			\end{itemize}
		\item Si no lo está:
			\begin{itemize}
				\item Se le redirige al \textit{login}.
			\end{itemize}
	\end{itemize}
\end{itemize}

\bigskip
Una vez implementado esto solo ha sido necesario indicar en el \textit{router} las rutas que se quieren proteger.


\subsection{Variables de entorno}


Para que la comunicación frontend-backend pueda llevarse a cabo es necesario definir una serie de variables de entorno. Estas indican la \textit{URL} en la que se encuentra la \textit{API}, entre otros. Son:
\begin{itemize}
	\item \textit{production}: Utilizada por \textit{Angular} a la hora de construir la aplicación.
	\item \textit{backendUrl}: \textit{URL} sin protocolo de la \textit{API}.
	\item \textit{httpsEnabled}: Indica si el protocolo de la \textit{API} es \textit{HTTP} o \textit{HTTPS}.
\end{itemize}

\bigskip
Actualmente el proyecto cuenta con cuatro entornos distintos. El primero es el de desarrollo, utilizado durante el desarrollo del proyecto. El segundo es el de producción, el cual es el que se debe usar a la hora de utilizar el proyecto en producción. El tercero, llamado \textit{on-premise} es el usado a la hora de construir la imagen de \textit{Docker} y está configurado por defecto para funcionar \textit{out-of-the-box} con el \textit{docker-compose}. El último, que se encuentra en el archivo \textit{env.j}s en la raíz del directorio \textit{src} del proyecto, permite redefinir las variables de entorno sin tener que reconstruir el backend. Por defecto se encuentra desactivado.


\subsection{Interfaces}


Se han creado además una serie de interfaces para controlar más aún los datos que se manejan en el frontend. Estas contemplan cada tipo de dato y son:
\begin{itemize}
	\item \textit{AccessToken}: \textit{Token} de acceso.
	\item \textit{BasicAuth}: Credenciales de usuario.
	\item \textit{Container}: Contenedor, usado en el módulo de despliegues.
	\item \textit{SingleContainerOperation}: Operación que se ejecuta en un único contenedor.
	\item \textit{ContainerOperation}: Operación que se ejecuta en todos los contenedores.
	\item \textit{Image}: Imagen, usada en el módulo de despliegues.
	\item \textit{SingleImageOperation}: Operación que se ejecuta en una única imagen.
	\item \textit{ImageOperation}: Operación que se ejecuta en todas las imágenes.
	\item \textit{Customer}: Cliente.
	\item \textit{DockerHubImage}: Imagen de \textit{DockerHub}, usada al utilizar la operación de búsqueda de imágenes.
	\item \textit{Host}: Grupo de \textit{hosts}.
	\item \textit{Machine}: Máquina.
	\item \textit{Playbook}: Playbook.
	\item \textit{Query}: Criterio de búsqueda, usado en todos los \textit{endpoints} para hacer consultas complejas.
	\item \textit{StatusResponse}: Estado del backend.
	\item \textit{User}: Usuario.
\end{itemize}


\subsection{Componentes}


En cuanto a componentes se ha intentado abstraer y reutilizar lo máximo posible.


\subsubsection{Tabla}


Usado para crear tablas dinánicas sin necesidad de configuraciones exahustivas. Sus funcionalidades extra son una barra de búsqueda de elementos, un selector de las columnas que se muestran, una barra inferior para paginación y la posibilidad de agregar un botón de acción personalizado para cada item de la tabla. 

\bigskip
Parámetros de entrada:
\begin{itemize}
	\item \textit{title}: Título de la tabla.
	\item \textit{displayedColumns}: Columnas que se quieren mostrar.
	\item \textit{deselectedColumns}: Columas que por defecto no aparecerán mostradas.
	\item \textit{data}: \textit{Array} con los datos a mostrar.
	\item \textit{actions}: \textit{Array} de strings para configurar las acciones disponibles para cada item.
	\item \textit{customActionData}: Diccionario con la configuración de la acción personalizada adicional.
\end{itemize}

Por defecto cada item tiene cuatro acciones asociadas: \textit{play} (P), \textit{detail} (D), \textit{edit} (E) y \textit{remove} (R). Estas se activan mediante el array \textit{actions}, agregando los idenficativos de las acciones que se deseen a este. En cambio, para configurar la acción adicional se debe incluir en \textit{customActionData} el icono y el tooltip a mostrar y agregar el identificativo 'C' a \textit{actions}.

Estas acciones son botones que se agregan en una columna de la tabla y cuando estos se pulsan emiten un evento con los datos del item al que pertenezcan. De este modo solo queda implementar la lógica en el componente que esté usando la tabla para que tome esos datos y los procese a voluntad.

Un ejemplo de uso de esta tabla sería el siguiente:

\begin{lstlisting}
<app-ipmtable
	*ngIf="data"
	[displayedColumns]="displayedColumns"
	[deselectedColumns]="['id']"
	[actions]="['P','C','R']"
	[data]="data"
	(playCallback)="runImage($event)"
	(removeCallback)="removeImage($event)"
	(customActionCallback)="manageImage($event)"
	[customActionData]="manageContainerCustomActionData"
>
\end{lstlisting}



\bigskip
Parámetros de salida:
\begin{itemize}
	\item \textit{playCallback}
	\item \textit{detailCallback}
	\item \textit{editCallback}
	\item \textit{removeCallback}
	\item \textit{customActionCallback}
\end{itemize}



\subsubsection{Navegador superior}

Se trata de una barra de navegación superior que cuenta con diferentes botones para navegar por las diferentes páginas. Cuenta con lógica interna para desactivar o no el enlace a la página de despliegues en caso de que \textit{Docker} no se encuentre activado y para mostrar o no el enlace a la administración de usuarios.

\bigskip
Los enlaces con los que cuenta son:
\begin{itemize}
	\item Home
	\item Despliegues
	\item Aprovisionamiento
	\item Máquinas
	\item Administración de usuarios
	\item Logout
\end{itemize}



\subsubsection{Home (/)}

Este componente muestra tarjetas con información sobre el estado del backend. Esta información depende del usuario que se encuentre visitandola, ya que un administrador recibirá más información que un usuario regular. Por otro lado también se adapta a la disponibilidad de \textit{Docker} en el backend.


\subsubsection{Login (/login)}


Componente bastante sencillo que cuenta con un formulario con tres campos. El primero es opcional y sirve para indicar el cliente al que se quiere conectar el usuario. El segundo y tercer campo son su usuario y contraseña.


\subsubsection{Admin (/admin)}
\label{sec:admin}

Administración de usuarios. Este componente solo es accesible por usuarios administradores. Hace uso de la tabla anterior para mostrar los usuarios que se encuentran registrados en el cliente actual y permite operar con ellos. Se ha creado un diálogo anexo a este componente para crear y editar estos usuarios y se puede acceder a él mediante las acciones de la tabla.



\subsubsection{Despliegues (/deploy)}

Este componente se compone de dos pestañas, \textit{Containers} e \textit{Images}.
\begin{itemize}
	\item Containers: En esta primera pestaña se muestran los contenedores que se encuentran en el backend (en cualquier estado) y se pueden administrar. Esta administración consiste en un diálogo en el que se pueden ejecutar diferentes operaciones sobre cada contenedor, son:
	\item Pausar y despausar
	\item Recargar
	\item Reiniciar
	\item Parar
	\item Forzar parada
	\item Cambiar nombre
	\item Obtener logs
\end{itemize}

\bigskip
Además cuenta con dos botones específicos para este componente en la esquina superior derecha. Estos son para ejecutar imágenes, lo que nos lleva a la segunda pestaña, y para eliminar los contenedores obsoletos (mediante una operación).
\begin{itemize}
	\item \textit{Images}: Aquí aparecen las imágenes que se encuentran también en el backend. Cada una de estas imágenes se puede ejecutar, administrar (recargar y obtener historial) y borrar del backend. Los botones de la esquina superior derecha son para hacer búsquedas de imágenes en \textit{DockerHub} y para eliminar aquellas imágenes obsoletas del sistema.
\end{itemize}




\subsubsection{Aprovisionamiento (/provision)}


En el caso del aprovisionamiento este componente cuenta con tres pestañas, las cuales son:
\begin{itemize}
	\item \textit{Playbooks}: Gestión de los \textit{playbooks}. Se pueden ejecutar, modificar y eliminar. Además cuenta con un botón para crear nuevos playbooks.
	\item \textit{Editor}. Editor de texto configurado con la sintaxis \textit{YAML} para crear y modificar los diferentes playbooks. Esta pestaña se encuentra enlazada con la primera ya que cuando se crea un playbook nuevo o se quiere modificar uno existente se redirige al usuario aquí.
	\item \textit{Host groups}: Gestión de los grupos de hosts de manera similar a los clientes o usuarios. Se encuentra enlazado con el servicio de máquinas para obtener las direcciones IP de las máquinas que se encuentran almacenadas en el sistema.
\end{itemize}

Del mismo modo que en otros componentes, se hace uso de diálogos para la gestión de los datos.


\subsubsection{Máquinas (/machines)}

Este componente es muy similar a \hyperref[sec:admin]{\textit{Admin}}. Cuenta con una tabla para mostrar los datos de todas las máquinas actuales y además se ha creado un diálogo (también accesible por las acciones) para crear y modificar los datos de las máquinas.


\subsubsection{AreYouSureDialog}


En ocasiones es necesaria es necesaria la confirmación del usuario a la hora de realizar una acción. Se ha creado este  diálogo para tener un componente común a todas estas confirmaciones. Al cerrarse emite un valor (\textit{true} o \textit{false}) que indica la decisión del usuario.



	
	\chapter{Conclusiones y trabajos futuros}

Tras el desarrollo de este proyecto se puede comprobar que los objetivos que se marcaron al inicio se han alcanzado exitosamente.

\textbf{IPManager} es una solución que responde al problema que se plantea ya que unifica los procesos de aprovisionado de sistemas, despliegue de servicios y almacenamiento de configuraciones de una manera sencilla y modular. Además, basada en una arquitectura de microservicios, provee de un backend y un frontend que pueden funcionar de forma independiente el uno del otro.

Es también una solución liviana y no intrusiva ya que se puede instalar en sistemas con pocos recursos de manera local o a través de contenedores \textit{Docker}, o incluso se puede desplegar en sistemas \textit{Cloud}.

Ofrece tambien una \textit{API REST} que posibilita las comunicaciones con otros sistemas. Un ejemplo de este uso es \textbf{IPMDroid}, una aplicación móvil para Android que se ha desarrollado de forma paralela a este proyecto para la asignatura ''Programación de Dispositivos Móviles''. Esta ofrece una interfaz sencilla para la administración de configuraciones y despliegue de servicios y se puede encontrar en \href{https://github.com/harvestcore/ipmdroid}{este repositorio} \cite{ipmdroid} en \textit{Github}.

Además, todo el proyecto está compuesto por software libre bajo la licencia \href{https://www.gnu.org/licenses/gpl-3.0.html}{GNU GPLv3} y se encuentra en \href{https://github.com/harvestcore/tfg}{este repositorio} \cite{ipmanager}.

Por último, y no por ello menos importante, se ha aprendido a desarrollar un proyecto de grandes dimensiones desde las etapas más tempranas, como son las primeras investigaciones e ideas, hasta las finales, dejando la puerta abierta a posibles mejoras y actualizaciones.


\pagebreak
En cuanto a trabajos futuros hay ciertos aspectos que se podrían mejorar:

\begin{itemize}
	\item Actualmente el módulo para despliegues de servicios solo permite realizar operaciones con imágenes y contenedores, por lo que una mejora sería integrar el manejo de redes, volúmenes u otras configuraciones relacionadas con \textit{Docker}.
	\item Se podrían ampliar los datos que se almacenan en el módulo de configuraciones. Las actuales se centran en configuraciones de red, pero podrían ampliarse para almacenar cualquier otro aspecto.
	\item El módulo de aprovisionamiento se centra exclusivamente en la ejecución de \textit{Playbooks}. Una mejora en este aspecto podría ser un mejor manejo de los parámetros de ejecución de estos.
	\item El frontend desarrollado intenta ser sencillo a la par que funcional, pero hay características que se podrían mejorar. Una mejora podría ser las vistas individuales de las configuraciones de las máquinas que se almacenan.
\end{itemize}


	
	\appendix
	\chapter{Especificación de \textit{endpoints}}

\textbf{Leyenda:}

\textbf{E/S}: Parámetro de \textbf{E}ntrada o \textbf{S}alida.


\section{Status}

\subsection{\textit{GET /status}}
Devuelve el estado actual de los dos servicios principales del backend, \textit{MongoDB} y \textit{Docker}.

Cabeceras necesarias:
\begin{table}[h!]
	\centering
	\begin{tabular}{|l|l|l|}
		\hline
		Nombre & Opcional & Descripción \\ \hline
		x-access-token & No & Token de acceso. \\ \hline
	\end{tabular}
\end{table}

Respuesta:
\begin{table}[h!]
	\centering
	\begin{adjustbox}{max width=\textwidth}
	\begin{tabular}{|l|l|l|l|}
		\hline
		Parámetro & Key & Tipo & Descripción \\ \hline
		mongo &  & dict &  \\ \hline
		& is\_up & bool & Si el servicio se encuentra activo o no. \\ \hline
		& data\_usage & list[dict] & Información de uso de datos de cada base de datos. \\ \hline
		& info & string & Información adicional del cliente de MongoDB. \\ \hline
		docker &  & dict &  \\ \hline
		& is\_up & bool & Si el servicio se encuentra activo o no. \\ \hline
		& data\_usage & dict & Información de uso de las imágenes y contenedores. \\ \hline
		& info & string & Información adicional del cliente de Docker. \\ \hline
	\end{tabular}
	\end{adjustbox}
\end{table}

\pagebreak
En caso de que el servicio de \textit{Docker} no se encuentre activado o presenta errores los \textit{endpoints} correspondientes al servicio de despliegue no estarán disponibles y el estado de este en la respuesta anterior tendrá la siguiente forma:

\begin{table}[h!]
	\centering
	
	\begin{adjustbox}{max width=\textwidth}
	\begin{tabular}{|l|l|l|l|}
		\hline
		Parámetro & Key & Tipo & Descripción \\ \hline
		docker &  & dict &  \\ \hline
		& status & bool & False. El servicio no funcina correctamente. \\ \hline
		& disabled & bool & Si el servicio se encuentra desactivado o no. \\ \hline
		& msg & string & Información adicional. \\ \hline
	\end{tabular}
	\end{adjustbox}
\end{table}






\section{Heartbeat}

\subsection{\textit{GET /api/heartbeat}}

Devuelve el estado actual de los dos servicios principales del backend de forma simplificada.

Respuesta:
\begin{table}[h!]
	\centering
\begin{adjustbox}{max width=\textwidth}
	\begin{tabular}{|l|l|l|}
		\hline
		Parámetro & Tipo & Descripción \\ \hline
		ok & bool & Si los servicios del sistema funcionan correctamente o no. \\ \hline
	\end{tabular}
\end{adjustbox}
\end{table}










\section{Autenticación}

\subsection{\textit{GET /login}}
Loguea al usuario en el cliente.

Cabeceras necesarias:
\begin{table}[h!]
	\centering
	\begin{adjustbox}{max width=\textwidth}
	\begin{tabular}{|l|l|l|}
		\hline
		Nombre & Opcional & Descripción \\ \hline
		x-access-token & No & Token de acceso. \\ \hline
	\end{tabular}
\end{adjustbox}
\end{table}


Respuesta:
\begin{table}[h!]
	\centering
	\begin{adjustbox}{max width=\textwidth}
	\begin{tabular}{|l|l|l|}
		\hline
		Parámetro & Tipo & Descripción \\ \hline
		token & string & Token JWT utilizado para identificar al usuario. \\ \hline
	\end{tabular}
\end{adjustbox}
\end{table}


\pagebreak
\subsection{\textit{POST /logout}}

Desloguea al usuario del cliente.

\bigskip
Cabeceras necesarias:

\begin{table}[h!]
	\centering
\begin{adjustbox}{max width=\textwidth}
	\begin{tabular}{|l|l|l|}
		\hline
		Nombre & Opcional & Descripción \\ \hline
		x-access-token & No & Token de acceso. \\ \hline
	\end{tabular}
\end{adjustbox}
\end{table}


Respuesta:
\begin{table}[h!]
	\centering
	\begin{adjustbox}{max width=\textwidth}
	\begin{tabular}{|l|l|l|}
		\hline
		Parámetro & Tipo & Descripción \\ \hline
		ok & bool & Si la operación se ha ejecutado correctamente o no. \\ \hline
		message & string & Mensaje complementario al estado de la operación. \\ \hline
	\end{tabular}
\end{adjustbox}
\end{table}




\section{Clientes}

\subsection{Cliente}
\label{sec:cliente}
\begin{table}[h!]
	\centering
	\begin{adjustbox}{max width=\textwidth}
	\begin{tabular}{|l|l|l|l|l|}
		\hline
		Parámetro & Tipo & Opcional & E/S & Descripción \\ \hline
		domain & string & No & E/S & Subdominio del cliente. \\ \hline
		db\_name & string & No & E/S & Base de datos del cliente. \\ \hline
	\end{tabular}
\end{adjustbox}
\end{table}


\subsection{\textit{POST /customer/query}}
Devuelve los clientes que cumplen los criterios de búsqueda.

Cabeceras necesarias:
\begin{table}[h!]
	\centering
	\begin{adjustbox}{max width=\textwidth}
	\begin{tabular}{|l|l|l|}
		\hline
		Nombre & Opcional & Descripción \\ \hline
		x-access-token & No & Token de acceso. \\ \hline
	\end{tabular}
\end{adjustbox}
\end{table}

Cuerpo de la petición (\textit{JSON}):
\begin{table}[h!]
	\centering
	\begin{adjustbox}{max width=\textwidth}
	\begin{tabular}{|l|l|l|l|}
		\hline
		Parámetro & Tipo & Opcional & Descripción \\ \hline
		query & dict & No & Criterio de búsqueda. \\ \hline
		filter & dict & Sí & Parámetros que se quieren en la respuesta. \\ \hline
	\end{tabular}
\end{adjustbox}
\end{table}

\pagebreak
Respuesta (un solo usuario):
\begin{table}[h!]
	\centering
	
	\begin{adjustbox}{max width=\textwidth}
	\begin{tabular}{|l|l|l|}
		\hline
		Parámetro & Tipo & Descripción \\ \hline
		data & \hyperref[sec:cliente]{Cliente} & Cliente que cumple el criterio de búsqueda. \\ \hline
	\end{tabular}
\end{adjustbox}
\end{table}

Respuesta (más de un usuario):
\begin{table}[h!]
	\centering
	\begin{adjustbox}{max width=\textwidth}
	\begin{tabular}{|l|l|l|}
		\hline
		Parámetro & Tipo & Descripción \\ \hline
		total & int & Número de clientes que cumplen el criterio de búsqueda. \\ \hline
		items & list[\hyperref[sec:cliente]{Cliente}] & Clientes que cumplen el criterio de búsqueda. \\ \hline
	\end{tabular}
\end{adjustbox}
\end{table}



\subsection{\textit{POST /customer}}
Crea un nuevo cliente.

Cabeceras necesarias:
\begin{table}[h!]
	\centering
	\begin{tabular}{|l|l|l|}
		\hline
		Nombre & Opcional & Descripción \\ \hline
		x-access-token & No & Token de acceso. \\ \hline
	\end{tabular}
\end{table}

Cuerpo de la petición (\textit{JSON}): \hyperref[sec:cliente]{Cliente}


Respuesta:
\begin{table}[h!]
	\centering
	\begin{adjustbox}{max width=\textwidth}
	\begin{tabular}{|l|l|l|}
		\hline
		Parámetro & Tipo & Descripción \\ \hline
		ok & bool & Si la operación se ha ejecutado correctamente o no. \\ \hline
		message & string & Mensaje complementario al estado de la operación. \\ \hline
	\end{tabular}
\end{adjustbox}
\end{table}



\subsection{\textit{PUT /customer}}
Modifica los datos de un cliente.

Cabeceras necesarias:
\begin{table}[h!]
	\centering
	\begin{adjustbox}{max width=\textwidth}
	\begin{tabular}{|l|l|l|}
		\hline
		Nombre & Opcional & Descripción \\ \hline
		x-access-token & No & Token de acceso. \\ \hline
	\end{tabular}
\end{adjustbox}
\end{table}

Cuerpo de la petición (\textit{JSON}):
\begin{table}[h!]
	\centering
	\begin{adjustbox}{max width=\textwidth}
	\begin{tabular}{|l|l|l|l|}
		\hline
		Parámetro & Tipo & Opcional & Descripción \\ \hline
		domain & string & No & Subominio del cliente que se quiere modificar. \\ \hline
		data & \hyperref[sec:cliente]{Cliente} & No & Nuevos datos del cliente. \\ \hline
	\end{tabular}
\end{adjustbox}
\end{table}

\pagebreak
Respuesta:
\begin{table}[h!]
	\centering
	\begin{adjustbox}{max width=\textwidth}
	\begin{tabular}{|l|l|l|}
		\hline
		Parámetro & Tipo & Descripción \\ \hline
		ok & bool & Si la operación se ha ejecutado correctamente o no. \\ \hline
		message & string & Mensaje complementario al estado de la operación. \\ \hline
	\end{tabular}
\end{adjustbox}
\end{table}






\subsection{\textit{DELETE /customer}}
Elimina un cliente.

Cabeceras necesarias:
\begin{table}[h!]
	\centering
	\begin{adjustbox}{max width=\textwidth}
	\begin{tabular}{|l|l|l|}
		\hline
		Nombre & Opcional & Descripción \\ \hline
		x-access-token & No & Token de acceso. \\ \hline
	\end{tabular}
\end{adjustbox}
\end{table}

Cuerpo de la petición (\textit{JSON}):
\begin{table}[h!]
	\centering
	\begin{adjustbox}{max width=\textwidth}
	\begin{tabular}{|l|l|l|l|}
		\hline
		Parámetro & Tipo & Opcional & Descripción \\ \hline
		domain & string & No & Subdominio del cliente que se quiere eliminar. \\ \hline
	\end{tabular}
\end{adjustbox}
\end{table}

Respuesta:
\begin{table}[h!]
	\centering
	\begin{adjustbox}{max width=\textwidth}
	\begin{tabular}{|l|l|l|}
		\hline
		Parámetro & Tipo & Descripción \\ \hline
		ok & bool & Si la operación se ha ejecutado correctamente o no. \\ \hline
		message & string & Mensaje complementario al estado de la operación. \\ \hline
	\end{tabular}
\end{adjustbox}
\end{table}


\section{Usuarios}

\subsection{Usuario}
\label{sec:usuario}
\begin{table}[h!]
	\centering
	\begin{adjustbox}{max width=\textwidth}
	\begin{tabular}{|l|l|l|l|l|}
		\hline
		Parámetro & Tipo & Opcional & E/S & Descripción \\ \hline
		type & string & No & E/S & Tipo de usuario. \\ \hline
		public\_id & string & - & S & UUID del usuario. \\ \hline
		first\_name & string & No & E/S & Nombre del usuario. \\ \hline
		last\_name & string & No & E/S & Apellido del usuario. \\ \hline
		username & string & No & E/S & Nickname del usuario. \\ \hline
		email & string & No & E/S & Email del usuario. \\ \hline
		password & string & No & E & Contraseña del usuario. \\ \hline
	\end{tabular}
\end{adjustbox}
\end{table}

\subsection{\textit{GET /user/:username}}
Devuelve toda la información asociada a un usuario.

Cabeceras necesarias:
\begin{table}[h!]
	\centering
	\begin{adjustbox}{max width=\textwidth}
	\begin{tabular}{|l|l|l|}
		\hline
		Nombre & Opcional & Descripción \\ \hline
		x-access-token & No & Token de acceso. \\ \hline
	\end{tabular}
\end{adjustbox}
\end{table}

Parámetros de la URL:
\begin{table}[h!]
	\centering
	\begin{adjustbox}{max width=\textwidth}
	\begin{tabular}{|l|l|l|}
		\hline
		Nombre & Opcional & Descripción \\ \hline
		username & No & Nombre del usuario a consultar. \\ \hline
	\end{tabular}
\end{adjustbox}
\end{table}

Respuesta:
\begin{table}[h!]
	\centering
	\begin{adjustbox}{max width=\textwidth}
	\begin{tabular}{|l|l|l|}
		\hline
		Parámetro & Tipo & Descripción \\ \hline
		data & \hyperref[sec:usuario]{Usuario} & Diccionario con toda la información del usuario consultado. \\ \hline
	\end{tabular}
\end{adjustbox}
\end{table}



\subsection{\textit{POST /user/query}}
Devuelve los usuarios que cumplen los criterios de búsqueda.

Cabeceras necesarias:
\begin{table}[h!]
	\centering
	\begin{adjustbox}{max width=\textwidth}
	\begin{tabular}{|l|l|l|}
		\hline
		Nombre & Opcional & Descripción \\ \hline
		x-access-token & No & Token de acceso. \\ \hline
	\end{tabular}
\end{adjustbox}
\end{table}


Cuerpo de la petición (\textit{JSON}):
\begin{table}[h!]
	\centering
	\begin{adjustbox}{max width=\textwidth}
	\begin{tabular}{|l|l|l|l|}
		\hline
		Parámetro & Tipo & Opcional & Descripción \\ \hline
		query & dict & No & Criterio de búsqueda. \\ \hline
		filter & dict & Sí & Parámetros que se quieren en la respuesta. \\ \hline
	\end{tabular}
\end{adjustbox}
\end{table}

Respuesta (un solo usuario):
\begin{table}[h!]
	\centering
	\begin{adjustbox}{max width=\textwidth}
	\begin{tabular}{|l|l|l|}
		\hline
		Parámetro & Tipo & Descripción \\ \hline
		data & \hyperref[sec:usuario]{Usuario} & Usuario que cumple el criterio de búsqueda. \\ \hline
	\end{tabular}
\end{adjustbox}
\end{table}

\pagebreak
Respuesta (más de un usuario):
\begin{table}[h!]
	\centering
	\begin{adjustbox}{max width=\textwidth}
	\begin{tabular}{|l|l|l|}
		\hline
		Parámetro & Tipo & Descripción \\ \hline
		total & int & Número de usuarios que cumplen el criterio de búsqueda. \\ \hline
		items & list[\hyperref[sec:usuario]{Usuario}] & Usuarios que cumplen el criterio de búsqueda. \\ \hline
	\end{tabular}
\end{adjustbox}
\end{table}

\subsection{\textit{POST /user}}
Crea un nuevo usuario.

Cabeceras necesarias:
\begin{table}[h!]
	\centering
	\begin{adjustbox}{max width=\textwidth}
	\begin{tabular}{|l|l|l|}
		\hline
		Nombre & Opcional & Descripción \\ \hline
		x-access-token & No & Token de acceso. \\ \hline
	\end{tabular}
\end{adjustbox}
\end{table}

Cuerpo de la petición (\textit{JSON}): \hyperref[sec:usuario]{Usuario}

Respuesta:
\begin{table}[h!]
	\centering
	\begin{adjustbox}{max width=\textwidth}
	\begin{tabular}{|l|l|l|}
		\hline
		Parámetro & Tipo & Descripción \\ \hline
		ok & bool & Si la operación se ha ejecutado correctamente o no. \\ \hline
		message & string & Mensaje complementario al estado de la operación. \\ \hline
	\end{tabular}
\end{adjustbox}
\end{table}




\subsection{\textit{PUT /user}}
Modifica los datos de un usuario.

Cabeceras necesarias:
\begin{table}[h!]
	\centering
	\begin{adjustbox}{max width=\textwidth}
	\begin{tabular}{|l|l|l|}
		\hline
		Nombre & Opcional & Descripción \\ \hline
		x-access-token & No & Token de acceso. \\ \hline
	\end{tabular}
\end{adjustbox}
\end{table}


Cuerpo de la petición (\textit{JSON}):
\begin{table}[h!]
	\centering
	\begin{adjustbox}{max width=\textwidth}
	\begin{tabular}{|l|l|l|l|}
		\hline
		Parámetro & Tipo & Opcional & Descripción \\ \hline
		email & string & No & Email del usuario que se quiere modificar. \\ \hline
		data & \hyperref[sec:usuario]{Usuario} & No & Nuevos datos del usuario. \\ \hline
	\end{tabular}
\end{adjustbox}
\end{table}

Respuesta:
\begin{table}[h!]
	\centering
	\begin{adjustbox}{max width=\textwidth}
	\begin{tabular}{|l|l|l|}
		\hline
		Parámetro & Tipo & Descripción \\ \hline
		ok & bool & Si la operación se ha ejecutado correctamente o no. \\ \hline
		message & string & Mensaje complementario al estado de la operación. \\ \hline
	\end{tabular}
\end{adjustbox}
\end{table}








\subsection{\textit{DELETE /user}}
Elimina un usuario.

Cabeceras necesarias:
\begin{table}[h!]
	\centering
	\begin{adjustbox}{max width=\textwidth}
	\begin{tabular}{|l|l|l|}
		\hline
		Nombre & Opcional & Descripción \\ \hline
		x-access-token & No & Token de acceso. \\ \hline
	\end{tabular}
\end{adjustbox}
\end{table}

Cuerpo de la petición (\textit{JSON}):
\begin{table}[h!]
	\centering
	\begin{adjustbox}{max width=\textwidth}
	\begin{tabular}{|l|l|l|l|}
		\hline
		Parámetro & Tipo & Opcional & Descripción \\ \hline
		email & string & No & Email del usuario que se quiere eliminar. \\ \hline
	\end{tabular}
\end{adjustbox}
\end{table}

Respuesta:
\begin{table}[h!]
	\centering
	\begin{adjustbox}{max width=\textwidth}
	\begin{tabular}{|l|l|l|}
		\hline
		Parámetro & Tipo & Descripción \\ \hline
		ok & bool & Si la operación se ha ejecutado correctamente o no. \\ \hline
		message & string & Mensaje complementario al estado de la operación. \\ \hline
	\end{tabular}
\end{adjustbox}
\end{table}





\section{Máquinas}

\subsection{Machine}
\label{sec:maquina}
\begin{table}[h!]
	\centering
	\begin{adjustbox}{max width=\textwidth}
	\begin{tabular}{|l|l|l|l|l|}
		\hline
		Parámetro & Tipo & Opcional & E/S & Descripción \\ \hline
		name & string & No & E/S & Tipo de usuario. \\ \hline
		description & string & Sí & E/S & Descripción de la máquina. \\ \hline
		type & string & No & E/S & Tipo de la máquina. \\ \hline
		ipv4 & string & Sí & E/S & Dirección IPv4 de la máquina. \\ \hline
		ipv6 & string & Sí & E/S & Dirección IPv6 de la máquina. \\ \hline
		mac & string & Sí & E/S & Dirección MAC de la máquina. \\ \hline
		broadcast & string & Sí & E/S & Broadcast de la red a la que se conecta la máquina. \\ \hline
		gateway & string & Sí & E/S & Gateway de la red a la que se conecta la máquina. \\ \hline
		netmask & string & Sí & E/S & Netmask de la red a la que se conecta la máquina. \\ \hline
		network & string & Sí & E/S & Network de la red a la que se conecta la máquina. \\ \hline
	\end{tabular}
\end{adjustbox}
\end{table}

\pagebreak
\subsection{\textit{GET /machine/:name}}
Devuelve toda la información asociada a una máquina.

Cabeceras necesarias:
\begin{table}[h!]
	\centering
	\begin{adjustbox}{max width=\textwidth}
	\begin{tabular}{|l|l|l|}
		\hline
		Nombre & Opcional & Descripción \\ \hline
		x-access-token & No & Token de acceso. \\ \hline
	\end{tabular}
\end{adjustbox}
\end{table}

Parámetros de la URL:
\begin{table}[h!]
	\centering
	\begin{adjustbox}{max width=\textwidth}
	\begin{tabular}{|l|l|l|}
		\hline
		Nombre & Opcional & Descripción \\ \hline
		name & No & Nombre de la máquina a consultar. \\ \hline
	\end{tabular}
\end{adjustbox}
\end{table}

Respuesta:
\begin{table}[h!]
	\centering
	\begin{adjustbox}{max width=\textwidth}
	\begin{tabular}{|l|l|l|}
		\hline
		Parámetro & Tipo & Descripción \\ \hline
		data & \hyperref[sec:maquina]{Machine} & Diccionario con toda la información de la máquina consultada. \\ \hline
	\end{tabular}
\end{adjustbox}
\end{table}



\subsection{\textit{POST/machine/query}}

Cabeceras necesarias:
\begin{table}[h!]
	\centering
	\begin{adjustbox}{max width=\textwidth}
	\begin{tabular}{|l|l|l|}
		\hline
		Nombre & Opcional & Descripción \\ \hline
		x-access-token & No & Token de acceso. \\ \hline
	\end{tabular}
\end{adjustbox}
\end{table}

Cuerpo de la petición (\textit{JSON}):
\begin{table}[h!]
	\centering
	\begin{adjustbox}{max width=\textwidth}
	\begin{tabular}{|l|l|l|l|}
		\hline
		Parámetro & Tipo & Opcional & Descripción \\ \hline
		query & dict & No & Criterio de búsqueda. \\ \hline
		filter & dict & Sí & Parámetros que se quieren en la respuesta. \\ \hline
	\end{tabular}
\end{adjustbox}
\end{table}

Respuesta (una sola máquina):
\begin{table}[h!]
	\centering
	\begin{adjustbox}{max width=\textwidth}
	\begin{tabular}{|l|l|l|}
		\hline
		Parámetro & Tipo & Descripción \\ \hline
		data & \hyperref[sec:maquina]{Machine} & Máquina que cumple el criterio de búsqueda. \\ \hline
	\end{tabular}
\end{adjustbox}
\end{table}

Respuesta (más de una máquina):
\begin{table}[h!]
	\centering
	\begin{adjustbox}{max width=\textwidth}
	\begin{tabular}{|l|l|l|}
		\hline
		Parámetro & Tipo & Descripción \\ \hline
		total & int & Número de máquinas que cumplen el criterio de búsqueda. \\ \hline
		items & list[\hyperref[sec:maquina]{Machine}] & Máquinas que cumplen el criterio de búsqueda. \\ \hline
	\end{tabular}
\end{adjustbox}
\end{table}



\subsection{\textit{POST /machine}}

Cabeceras necesarias:
\begin{table}[h!]
	\centering
	\begin{adjustbox}{max width=\textwidth}
	\begin{tabular}{|l|l|l|}
		\hline
		Nombre & Opcional & Descripción \\ \hline
		x-access-token & No & Token de acceso. \\ \hline
	\end{tabular}
\end{adjustbox}
\end{table}

Cuerpo de la petición (\textit{JSON}): \hyperref[sec:maquina]{Machine}

Respuesta:
\begin{table}[h!]
	\centering
	\begin{adjustbox}{max width=\textwidth}
	\begin{tabular}{|l|l|l|}
		\hline
		Parámetro & Tipo & Descripción \\ \hline
		ok & bool & Si la operación se ha ejecutado correctamente o no. \\ \hline
		message & string & Mensaje complementario al estado de la operación. \\ \hline
	\end{tabular}
\end{adjustbox}
\end{table}


\subsection{\textit{PUT /machine}}

Cabeceras necesarias:
\begin{table}[h!]
	\centering
	\begin{adjustbox}{max width=\textwidth}
	\begin{tabular}{|l|l|l|}
		\hline
		Nombre & Opcional & Descripción \\ \hline
		x-access-token & No & Token de acceso. \\ \hline
	\end{tabular}
\end{adjustbox}
\end{table}

Cuerpo de la petición (\textit{JSON}):
\begin{table}[h!]
	\centering
	\begin{adjustbox}{max width=\textwidth}
	\begin{tabular}{|l|l|l|l|}
		\hline
		Parámetro & Tipo & Opcional & Descripción \\ \hline
		name & string & No & Nombre de la máquina que se quiere modificar. \\ \hline
		data & \hyperref[sec:maquina]{Machine} & No & Nuevos datos de la máquina. \\ \hline
	\end{tabular}
\end{adjustbox}
\end{table}


Respuesta:
\begin{table}[h!]
	\centering
	\begin{adjustbox}{max width=\textwidth}
	\begin{tabular}{|l|l|l|}
		\hline
		Parámetro & Tipo & Descripción \\ \hline
		ok & bool & Si la operación se ha ejecutado correctamente o no. \\ \hline
		message & string & Mensaje complementario al estado de la operación. \\ \hline
	\end{tabular}
\end{adjustbox}
\end{table}


\subsection{\textit{DELETE /machine}}

Cabeceras necesarias:
\begin{table}[h!]
	\centering
	\begin{adjustbox}{max width=\textwidth}
	\begin{tabular}{|l|l|l|}
		\hline
		Nombre & Opcional & Descripción \\ \hline
		x-access-token & No & Token de acceso. \\ \hline
	\end{tabular}
\end{adjustbox}
\end{table}

\pagebreak

Cuerpo de la petición (\textit{JSON}):
\begin{table}[h!]
	\centering
	\begin{adjustbox}{max width=\textwidth}
	\begin{tabular}{|l|l|l|l|}
		\hline
		Parámetro & Tipo & Opcional & Descripción \\ \hline
		name & string & No & Nombre de la máquina que se quiere eliminar. \\ \hline
	\end{tabular}
\end{adjustbox}
\end{table}

Respuesta:
\begin{table}[h!]
	\centering
	\begin{adjustbox}{max width=\textwidth}
	\begin{tabular}{|l|l|l|}
		\hline
		Parámetro & Tipo & Descripción \\ \hline
		ok & bool & Si la operación se ha ejecutado correctamente o no. \\ \hline
		message & string & Mensaje complementario al estado de la operación. \\ \hline
	\end{tabular}
\end{adjustbox}
\end{table}










\section{Grupos de hosts}

\subsection{\textit{Hosts}}
\label{sec:hosts}
\begin{table}[h!]
	\centering
	\begin{adjustbox}{max width=\textwidth}
	\begin{tabular}{|l|l|l|l|l|}
		\hline
		Parámetro & Tipo & Opcional & E/S & Descripción \\ \hline
		name & string & No & E/S & Nombre del grupo de hosts. \\ \hline
		ips & list[string] & No & E/S & Direcciones IPv4. \\ \hline
	\end{tabular}
\end{adjustbox}
\end{table}

\subsection{\textit{GET /provision/hosts/:name}}

Devuelve toda la información asociada a un grupo de \textit{hosts}.

Cabeceras necesarias:

\begin{table}[h!]
	\centering
	\begin{adjustbox}{max width=\textwidth}
	\begin{tabular}{|l|l|l|}
		\hline
		Nombre & Opcional & Descripción \\ \hline
		x-access-token & No & Token de acceso. \\ \hline
	\end{tabular}
\end{adjustbox}
\end{table}

Parámetros de la URL:

\begin{table}[h!]
	\centering
	\begin{adjustbox}{max width=\textwidth}
	\begin{tabular}{|l|l|l|}
		\hline
		Nombre & Opcional & Descripción \\ \hline
		name & No & Nombre del grupo de hosts a consultar. \\ \hline
	\end{tabular}
\end{adjustbox}
\end{table}

Respuesta:

\begin{table}[h!]
	\centering
	\begin{adjustbox}{max width=\textwidth}
	\begin{tabular}{|l|l|l|}
		\hline
		Parámetro & Tipo & Descripción \\ \hline
		data & \hyperref[sec:hosts]{Hosts} & Diccionario con toda la información del grupo de hosts consultado. \\ \hline
	\end{tabular}
\end{adjustbox}
\end{table}


\pagebreak
\subsection{\textit{POST/provision/hosts/query}}

Devuelve los grupos de \textit{hosts} que cumplen los criterios de búsqueda.

Cabeceras necesarias:

\begin{table}[h!]
	\centering
	\begin{adjustbox}{max width=\textwidth}
	\begin{tabular}{|l|l|l|}
		\hline
		Nombre & Opcional & Descripción \\ \hline
		x-access-token & No & Token de acceso. \\ \hline
	\end{tabular}
\end{adjustbox}
\end{table}

Cuerpo de la petición (\textit{JSON}):
\begin{table}[h!]
	\centering
	\begin{adjustbox}{max width=\textwidth}
	\begin{tabular}{|l|l|l|l|}
		\hline
		Parámetro & Tipo & Opcional & Descripción \\ \hline
		query & dict & No & Criterio de búsqueda. \\ \hline
		filter & dict & Sí & Parámetros que se quieren en la respuesta. \\ \hline
	\end{tabular}
\end{adjustbox}
\end{table}

Respuesta (una sola máquina):
\begin{table}[h!]
	\centering
	\begin{adjustbox}{max width=\textwidth}
	\begin{tabular}{|l|l|l|}
		\hline
		Parámetro & Tipo & Descripción \\ \hline
		data & \hyperref[sec:hosts]{Hosts} & Grupo de hosts que cumple el criterio de búsqueda. \\ \hline
	\end{tabular}
\end{adjustbox}
\end{table}

Respuesta (más de una máquina):
\begin{table}[h!]
	\centering
	\begin{adjustbox}{max width=\textwidth}
	\begin{tabular}{|l|l|l|}
		\hline
		Parámetro & Tipo & Descripción \\ \hline
		total & int & Número de grupos de hosts que cumplen el criterio de búsqueda. \\ \hline
		items & list[\hyperref[sec:hosts]{Hosts}] & Grupos de hosts que cumplen el criterio de búsqueda. \\ \hline
	\end{tabular}
\end{adjustbox}
\end{table}



\subsection{\textit{POST /provision/hosts}}

Cabeceras necesarias:
\begin{table}[h!]
	\centering
	\begin{adjustbox}{max width=\textwidth}
	\begin{tabular}{|l|l|l|}
		\hline
		Nombre & Opcional & Descripción \\ \hline
		x-access-token & No & Token de acceso. \\ \hline
	\end{tabular}
\end{adjustbox}
\end{table}

Cuerpo de la petición (\textit{JSON}): \hyperref[sec:hosts]{Hosts}

Respuesta:
\begin{table}[h!]
	\centering
	\begin{adjustbox}{max width=\textwidth}
	\begin{tabular}{|l|l|l|}
		\hline
		Parámetro & Tipo & Descripción \\ \hline
		ok & bool & Si la operación se ha ejecutado correctamente o no. \\ \hline
		message & string & Mensaje complementario al estado de la operación. \\ \hline
	\end{tabular}
\end{adjustbox}
\end{table}

\pagebreak
\subsection{\textit{PUT /provision/hosts}}

Cabeceras necesarias:
\begin{table}[h!]
	\centering
	\begin{adjustbox}{max width=\textwidth}
	\begin{tabular}{|l|l|l|}
		\hline
		Nombre & Opcional & Descripción \\ \hline
		x-access-token & No & Token de acceso. \\ \hline
	\end{tabular}
\end{adjustbox}
\end{table}

Cuerpo de la petición (\textit{JSON}):
\begin{table}[h!]
	\centering
	\begin{adjustbox}{max width=\textwidth}
	\begin{tabular}{|l|l|l|l|}
		\hline
		Parámetro & Tipo & Opcional & Descripción \\ \hline
		name & string & No & Nombre del grupo de hosts que se quiere modificar. \\ \hline
		data & \hyperref[sec:hosts]{Hosts} & No & Nuevos datos del grupo de hosts. \\ \hline
	\end{tabular}
\end{adjustbox}
\end{table}


Respuesta:
\begin{table}[h!]
	\centering
	\begin{adjustbox}{max width=\textwidth}
	\begin{tabular}{|l|l|l|}
		\hline
		Parámetro & Tipo & Descripción \\ \hline
		ok & bool & Si la operación se ha ejecutado correctamente o no. \\ \hline
		message & string & Mensaje complementario al estado de la operación. \\ \hline
	\end{tabular}
\end{adjustbox}
\end{table}

\subsection{\textit{DELETE /provision/hosts}}

Cabeceras necesarias:
\begin{table}[h!]
	\centering
	\begin{adjustbox}{max width=\textwidth}
	\begin{tabular}{|l|l|l|}
		\hline
		Nombre & Opcional & Descripción \\ \hline
		x-access-token & No & Token de acceso. \\ \hline
	\end{tabular}
\end{adjustbox}
\end{table}

Cuerpo de la petición (\textit{JSON}):
\begin{table}[h!]
	\centering
	\begin{adjustbox}{max width=\textwidth}
	\begin{tabular}{|l|l|l|l|}
		\hline
		Parámetro & Tipo & Opcional & Descripción \\ \hline
		name & string & No & Nombre del grupo de hosts que se quiere eliminar. \\ \hline
	\end{tabular}
\end{adjustbox}
\end{table}

Respuesta:
\begin{table}[h!]
	\centering
	\begin{adjustbox}{max width=\textwidth}
	\begin{tabular}{|l|l|l|}
		\hline
		Parámetro & Tipo & Descripción \\ \hline
		ok & bool & Si la operación se ha ejecutado correctamente o no. \\ \hline
		message & string & Mensaje complementario al estado de la operación. \\ \hline
	\end{tabular}
\end{adjustbox}
\end{table}









\pagebreak
\section{Playbooks}

	\subsection{Playbook}
	\label{sec:playbook}
		\begin{table}[h!]
			\centering
	\begin{adjustbox}{max width=\textwidth}
			\begin{tabular}{|l|l|l|l|l|}
				\hline
				Parámetro & Tipo & Opcional & E/S & Descripción \\ \hline
				name & string & No & E/S & Nombre del Playbook. \\ \hline
				playbook & dict & No & E/S & Contenido del Playbook codificado como \textit{JSON}. \\ \hline
			\end{tabular}
\end{adjustbox}
		\end{table}
	
	\subsection{\textit{GET /provision/playbook/:name}}
		Devuelve toda la información asociada a un Playbook.
		
		Cabeceras necesarias:
		\begin{table}[h!]
			\centering
	\begin{adjustbox}{max width=\textwidth}
			\begin{tabular}{|l|l|l|}
				\hline
				Nombre & Opcional & Descripción \\ \hline
				x-access-token & No & Token de acceso. \\ \hline
			\end{tabular}
\end{adjustbox}
		\end{table}
		
		Parámetros de la URL:
		\begin{table}[h!]
			\centering
	\begin{adjustbox}{max width=\textwidth}
			\begin{tabular}{|l|l|l|}
				\hline
				Nombre & Opcional & Descripción \\ \hline
				name & No & Nombre del Playbook a consultar. \\ \hline
			\end{tabular}
\end{adjustbox}
		\end{table}
		
		Respuesta:
		\begin{table}[h!]
			\centering
	\begin{adjustbox}{max width=\textwidth}
			\begin{tabular}{|l|l|l|}
				\hline
				Parámetro & Tipo & Descripción \\ \hline
				data & \hyperref[sec:hosts]{Hosts} & Diccionario con toda la información del Playbook consultado. \\ \hline
			\end{tabular}
\end{adjustbox}
		\end{table}
	
	
	
	\subsection{\textit{POST/provision/playbook/query}}
		Devuelve los \textit{playbooks} que cumplen los criterios de búsqueda.
		
		Cabeceras necesarias:
		\begin{table}[h!]
			\centering
	\begin{adjustbox}{max width=\textwidth}
			\begin{tabular}{|l|l|l|}
				\hline
				Nombre & Opcional & Descripción \\ \hline
				x-access-token & No & Token de acceso. \\ \hline
			\end{tabular}
\end{adjustbox}
		\end{table}
		
		Cuerpo de la petición (\textit{JSON}):
		\begin{table}[h!]
			\centering
	\begin{adjustbox}{max width=\textwidth}
			\begin{tabular}{|l|l|l|l|}
				\hline
				Parámetro & Tipo & Opcional & Descripción \\ \hline
				query & dict & No & Criterio de búsqueda. \\ \hline
				filter & dict & Sí & Parámetros que se quieren en la respuesta. \\ \hline
			\end{tabular}
\end{adjustbox}
		\end{table}
		
		
		\pagebreak
		Respuesta (un solo \textit{Playbook}):
		\begin{table}[h!]
			\centering
	\begin{adjustbox}{max width=\textwidth}
			\begin{tabular}{|l|l|l|}
				\hline
				Parámetro & Tipo & Descripción \\ \hline
				data & \hyperref[sec:playbook]{Playbook} & Playbook que cumple el criterio de búsqueda. \\ \hline
			\end{tabular}
\end{adjustbox}
		\end{table}
		
		Respuesta (más de un \textit{Playbook}):
		\begin{table}[h!]
			\centering
	\begin{adjustbox}{max width=\textwidth}
			\begin{tabular}{|l|l|l|}
				\hline
				Parámetro & Tipo & Descripción \\ \hline
				total & int & Número de Playbooks que cumplen el criterio de búsqueda. \\ \hline
				items & list[\hyperref[sec:playbook]{Playbook}] & Playbooks que cumplen el criterio de búsqueda. \\ \hline
			\end{tabular}
\end{adjustbox}
		\end{table}
	
	
	
	\subsection{\textit{POST /provision/playbook}}
		Crea un nuevo \textit{Playbook}.
		
		Cabeceras necesarias:
		\begin{table}[h!]
			\centering
	\begin{adjustbox}{max width=\textwidth}
			\begin{tabular}{|l|l|l|}
				\hline
				Nombre & Opcional & Descripción \\ \hline
				x-access-token & No & Token de acceso. \\ \hline
			\end{tabular}
\end{adjustbox}
		\end{table}
		
		Cuerpo de la petición (\textit{JSON}): \hyperref[sec:playbook]{Playbook}
		
		Respuesta:
		\begin{table}[h!]
			\centering
	\begin{adjustbox}{max width=\textwidth}
			\begin{tabular}{|l|l|l|}
				\hline
				Parámetro & Tipo & Descripción \\ \hline
				ok & bool & Si la operación se ha ejecutado correctamente o no. \\ \hline
				message & string & Mensaje complementario al estado de la operación. \\ \hline
			\end{tabular}
\end{adjustbox}
	\end{table}
	
	
	\subsection{\textit{PUT /provision/playbook}}
		Modifica los datos de un \textit{Playbook}.
		
		Cabeceras necesarias:
		\begin{table}[h!]
			\centering
	\begin{adjustbox}{max width=\textwidth}
			\begin{tabular}{|l|l|l|}
				\hline
				Nombre & Opcional & Descripción \\ \hline
				x-access-token & No & Token de acceso. \\ \hline
			\end{tabular}
\end{adjustbox}
		\end{table}
		
		Cuerpo de la petición (\textit{JSON}):
		\begin{table}[h!]
			\centering
	\begin{adjustbox}{max width=\textwidth}
			\begin{tabular}{|l|l|l|l|}
				\hline
				Parámetro & Tipo & Opcional & Descripción \\ \hline
				name & string & No & Nombre del grupo de hosts que se quiere modificar. \\ \hline
				data & \hyperref[sec:playbook]{Playbook} & No & Nuevos datos del grupo de hosts. \\ \hline
			\end{tabular}
\end{adjustbox}
		\end{table}
		
		
		\pagebreak
		Respuesta:
		\begin{table}[h!]
			\centering
	\begin{adjustbox}{max width=\textwidth}
			\begin{tabular}{|l|l|l|}
				\hline
				Parámetro & Tipo & Descripción \\ \hline
				ok & bool & Si la operación se ha ejecutado correctamente o no. \\ \hline
				message & string & Mensaje complementario al estado de la operación. \\ \hline
			\end{tabular}
\end{adjustbox}
		\end{table}
	
	\subsection{\textit{DELETE /provision/playbook}}
		Elimina un \textit{Playbook}.
		
		Cabeceras necesarias:
		\begin{table}[h!]
			\centering
	\begin{adjustbox}{max width=\textwidth}
			\begin{tabular}{|l|l|l|}
				\hline
				Nombre & Opcional & Descripción \\ \hline
				x-access-token & No & Token de acceso. \\ \hline
			\end{tabular}
\end{adjustbox}
		\end{table}
		
		Cuerpo de la petición (\textit{JSON}):
		\begin{table}[h!]
			\centering
	\begin{adjustbox}{max width=\textwidth}
			\begin{tabular}{|l|l|l|l|}
				\hline
				Parámetro & Tipo & Opcional & Descripción \\ \hline
				name & string & No & Nombre del Playbook que se quiere eliminar. \\ \hline
			\end{tabular}
\end{adjustbox}
		\end{table}
		
		Respuesta:
		\begin{table}[h!]
			\centering
	\begin{adjustbox}{max width=\textwidth}
			\begin{tabular}{|l|l|l|}
				\hline
				Parámetro & Tipo & Descripción \\ \hline
				ok & bool & Si la operación se ha ejecutado correctamente o no. \\ \hline
				message & string & Mensaje complementario al estado de la operación. \\ \hline
			\end{tabular}
\end{adjustbox}
		\end{table}











\section{Aprovisionamiento}


	\subsection{\textit{POST /provision}}
		Ejecuta un \textit{Playbook}.
	
		Cabeceras necesarias:
		\begin{table}[h!]
			\centering
	\begin{adjustbox}{max width=\textwidth}
			\begin{tabular}{|l|l|l|}
				\hline
				Nombre & Opcional & Descripción \\ \hline
				x-access-token & No & Token de acceso. \\ \hline
			\end{tabular}
\end{adjustbox}
		\end{table}
		
		\pagebreak
		Cuerpo de la petición (\textit{JSON}):
		\begin{table}[h!]
			\centering
	\begin{adjustbox}{max width=\textwidth}
			\begin{tabular}{|l|l|l|l|l|}
				\hline
				Parámetro & Key & Tipo & Opcional & Descripción \\ \hline
				hosts &  & list[string] & No & Lista de grupo de hosts donde se quiere ejecutar el playbook. \\ \hline
				playbook &  & string & No & Nombre del playbook a ejecutar. \\ \hline
				passwords &  & dict & No & Contraseñas necesarias para la conexión a los hosts. \\ \hline
				& conn\_pass & string & Sí & Contraseña de acceso. \\ \hline
				& become\_pass & string & Sí & Contraseña para acceder al root. \\ \hline
			\end{tabular}
\end{adjustbox}
		\end{table}
		
		Respuesta:
		\begin{table}[h!]
			\centering
	\begin{adjustbox}{max width=\textwidth}
			\begin{tabular}{|l|l|l|}
				\hline
				Parámetro & Tipo & Descripción \\ \hline
				result & string & Respuesta de la ejecución del playbook. \\ \hline
			\end{tabular}
\end{adjustbox}
		\end{table}







\section{Despliegue}

	\subsection{Contenedor}
	\label{sec:contenedor}
		\begin{table}[h!]
			\centering
	\begin{adjustbox}{max width=\textwidth}
			\begin{tabular}{|l|l|l|l|}
				\hline
				Parámetro & Tipo & E/S & Descripción \\ \hline
				id & string & S & Identificador del contenedor \\ \hline
				short\_id & string & S & Identificador del contenedor truncado a 10 carácteres. \\ \hline
				name & string & S & Nombre del contenedor. \\ \hline
				labels & dict & S & Etiquetas del contenedor. \\ \hline
				status & string & S & Estado del contenedor. \\ \hline
				image & Image & S & Imagen que se esta ejecutando en el contenedor. \\ \hline
			\end{tabular}
\end{adjustbox}
		\end{table}
	
	\subsection{Imagen}
	\label{sec:imagen}
		\begin{table}[h!]
			\centering
	\begin{adjustbox}{max width=\textwidth}
			\begin{tabular}{|l|l|l|l|}
				\hline
				Parámetro & Tipo & E/S & Descripción \\ \hline
				id & string & S & Identificador de la imagen. \\ \hline
				labels & dict & S & Etiquetas de la imagen. \\ \hline
				short\_id & string & S & Identificador de la imagen truncado a 10 carácteres. \\ \hline
				tags & list[string] & S & Tags de la imagen. \\ \hline
			\end{tabular}
\end{adjustbox}
		\end{table}
	
	\subsection{Filtro de contenedores}
	\label{sec:filtrocontenedor}
		\begin{table}[h!]
			\centering
	\begin{adjustbox}{max width=\textwidth}
			\begin{tabular}{|l|l|l|l|l|}
				\hline
				Parámetro & Key & Tipo & Opcional & Descripción \\ \hline
				filters &  & dict & No &  \\ \hline
				& exited & boolean & Sí & Si el contenedor ha finalizado su ejecución o no. \\ \hline
				& status & string & Sí & Estado del contenedor. \\ \hline
				& id & string & Sí & Identificador del contenedor. \\ \hline
				& name & string & Sí & Nombre del contenedor. \\ \hline
			\end{tabular}
\end{adjustbox}
		\end{table}
	
	\subsection{Filtro de imágenes}
	\label{sec:filtroimagen}
		\begin{table}[h!]
			\centering
	\begin{adjustbox}{max width=\textwidth}
			\begin{tabular}{|l|l|l|l|l|}
				\hline
				Parámetro & Key & Tipo & Opcional & Descripción \\ \hline
				filters &  & dict & No &  \\ \hline
				& dangling & boolean & Sí & Si la imagen se encuentra colgada o no. \\ \hline
				& label & string & Sí & Etiqueta de la imagen. \\ \hline
			\end{tabular}
\end{adjustbox}
		\end{table}
	
	
	\subsection{\textit{POST /deploy/container}}
		Permite ejecutar operaciones básicas en todos los contenedores.
		
		Cabeceras necesarias:
		\begin{table}[h!]
			\centering
	\begin{adjustbox}{max width=\textwidth}
			\begin{tabular}{|l|l|l|}
				\hline
				Nombre & Opcional & Descripción \\ \hline
				x-access-token & No & Token de acceso. \\ \hline
			\end{tabular}
\end{adjustbox}
		\end{table}
		
		Cuerpo de la petición (\textit{JSON}):
		\begin{table}[h!]
			\centering
	\begin{adjustbox}{max width=\textwidth}
			\begin{tabular}{|l|l|l|l|}
				\hline
				Parámetro & Tipo & Opcional & Descripción \\ \hline
				operation & string & No & Nombre de la operación que se quiere ejecutar. Valores posibles: \textit{run}, \textit{get}, \textit{list}, \textit{prune}. \\ \hline
				data & dict & No & Argumentos de la operación. \\ \hline
			\end{tabular}
\end{adjustbox}
		\end{table}
	
		Tipos de operaciones:
		\subsubsection{run}
			\begin{table}[h!]
				\centering
	\begin{adjustbox}{max width=\textwidth}
				\begin{tabular}{|l|l|l|l|l|}
					\hline
					Parámetro & Key & Tipo & Opcional & Descripción \\ \hline
					data &  & dict & No & Argumentos de la operación. \\ \hline
					& image & string & No & Imagen a ejecutar. \\ \hline
					& command & list[string] & Sí & Comando a ejecutar en el contenedor. \\ \hline
					& auto\_remove & bool & Sí & Eliminar o no el contenedor al terminar la ejecución. \\ \hline
					& detach & bool & Sí & Ejecutar o no el contenedor en segundo plano. Por defecto \textit{true}. \\ \hline
					& entrypoint & list[string] & Sí & Entrypoint del contenedor. \\ \hline
					& environment & dict & Sí & Variables de entorno. \\ \hline
					& hostname & string & Sí & Hostname del contenedor. \\ \hline
					& mounts & list[string] & Sí & Lista de volumenes que se montan en el contenedor. \\ \hline
					& name & string & Sí & Nombre del contenedor. \\ \hline
					& network & string & Sí & Nombre de la red a la que se conecta. \\ \hline
					& ports & dict & Sí & Puertos a enlazar. \\ \hline
					& user & string & Sí & Usuario para ejecutar los posibles comandos dentro del contenedor. \\ \hline
					& volumes & dict & Sí & Volumenes a montar. \\ \hline
					& working\_dir & string & Sí & Directorio de trabajo. \\ \hline
					& remove & bool & Sí & Eliminar el contenedor al terminar la ejecución. \\ \hline
				\end{tabular}
\end{adjustbox}
			\end{table}
		
		\pagebreak
		\subsubsection{get}
			\begin{table}[h!]
				\centering
	\begin{adjustbox}{max width=\textwidth}
				\begin{tabular}{|l|l|l|l|l|}
					\hline
					Parámetro & Key & Tipo & Opcional & Descripción \\ \hline
					data &  & dict & No & Argumentos de la operación. \\ \hline
					& container\_id & string & No & Identificador del contenedor \\ \hline
				\end{tabular}
\end{adjustbox}
			\end{table}
		
			Respuesta: \hyperref[sec:contenedor]{Contenedor}
			
		\subsubsection{list}
			\begin{table}[h!]
				\centering
	\begin{adjustbox}{max width=\textwidth}
				\begin{tabular}{|l|l|l|l|l|}
					\hline
					Parámetro & Key & Tipo & Opcional & Descripción \\ \hline
					data &  & dict & No & Argumentos de la operación. \\ \hline
					& all & bool & No & Mostrar o no todos los contenedores (en ejecución y detenidos o finalizados). \\ \hline
					& since & string & No & Mostrar contenedores creados desde este identificador. \\ \hline
					& before & string & No & Mostrar contenedores creados previos a este identificador. \\ \hline
					& filters & \hyperref[sec:filtrocontenedor]{Filtro} & No & Filtros para afinar la búsqueda. \\ \hline
				\end{tabular}
\end{adjustbox}
			\end{table}
		
			Respuesta (un solo contenedor):
			\begin{table}[h!]
				\centering
	\begin{adjustbox}{max width=\textwidth}
				\begin{tabular}{|l|l|l|}
					\hline
					Parámetro & Tipo & Descripción \\ \hline
					data & \hyperref[sec:contenedor]{Contenedor} & Contenedor que cumple el criterio de búsqueda. \\ \hline
				\end{tabular}
\end{adjustbox}
			\end{table}
		
			Respuesta (más de un contenedor):
			\begin{table}[h!]
				\centering
	\begin{adjustbox}{max width=\textwidth}
				\begin{tabular}{|l|l|l|}
					\hline
					Parámetro & Tipo & Descripción \\ \hline
					total & int & Número de contenedores que cumplen el criterio de búsqueda. \\ \hline
					items & list[\hyperref[sec:contenedor]{Contenedor}] & Contenedores que cumplen el criterio de búsqueda. \\ \hline
				\end{tabular}
\end{adjustbox}
			\end{table}
		
		\subsubsection{prune}
			\begin{table}[h!]
				\centering
	\begin{adjustbox}{max width=\textwidth}
				\begin{tabular}{|l|l|l|l|l|}
					\hline
					Parámetro & Key & Tipo & Opcional & Descripción \\ \hline
					data &  & dict & No & Argumentos de la operación. \\ \hline
					& filters & \hyperref[sec:filtrocontenedor]{Filtro} & Sí & Filtros. \\ \hline
				\end{tabular}
\end{adjustbox}
			\end{table}
	
	
	

	\subsection{\textit{POST /deploy/container/single}}
		Permite ejecutar operaciones básicas en un contenedor en concreto.
	
		Cabeceras necesarias:
		\begin{table}[h!]
			\centering
	\begin{adjustbox}{max width=\textwidth}
			\begin{tabular}{|l|l|l|}
				\hline
				Nombre & Opcional & Descripción \\ \hline
				x-access-token & No & Token de acceso. \\ \hline
			\end{tabular}
\end{adjustbox}
		\end{table}
		
		\pagebreak
		Cuerpo de la petición (\textit{JSON}):
		\begin{table}[h!]
			\centering
	\begin{adjustbox}{max width=\textwidth}
			\begin{tabular}{|l|l|l|l|}
				\hline
				Parámetro & Tipo & Opcional & Descripción \\ \hline
				container\_id & string & No & Identificador del contenedor. \\ \hline
				operation & string & No & Nombre de la operación que se quiere ejecutar. Valores posibles: \textit{kill}, \textit{logs}, \textit{pause}, \textit{reload}, \textit{rename}, \textit{restart}, \textit{stop}, \textit{unpause}. \\ \hline
				data & dict & No & Argumentos de la operación. \\ \hline
			\end{tabular}
\end{adjustbox}
		\end{table}
	
		Tipos de operaciones:
		
		\subsubsection{rename}
			\begin{table}[h!]
				\centering
	\begin{adjustbox}{max width=\textwidth}
				\begin{tabular}{|l|l|l|l|l|}
					\hline
					Parámetro & Key & Tipo & Opcional & Descripción \\ \hline
					data &  & dict & No & Argumentos de la operación. \\ \hline
					& name & string & No & Nuevo nombre del contenedor. \\ \hline
				\end{tabular}
\end{adjustbox}
			\end{table}
		
		\subsubsection{logs}
			Respuesta:
			
			\begin{table}[h!]
				\centering
	\begin{adjustbox}{max width=\textwidth}
				\begin{tabular}{|l|l|l|}
					\hline
					Parámetro & Tipo & Descripción \\ \hline
					data & string & Logs del contenedor. \\ \hline
				\end{tabular}
\end{adjustbox}
			\end{table}
		
	
	
	
	
	\subsection{\textit{POST /deploy/image}}
		Permite ejecutar operaciones básicas en todos las imágenes.
	
		Cabeceras necesarias:
		\begin{table}[h!]
			\centering
	\begin{adjustbox}{max width=\textwidth}
			\begin{tabular}{|l|l|l|}
				\hline
				Nombre & Opcional & Descripción \\ \hline
				x-access-token & No & Token de acceso. \\ \hline
			\end{tabular}
\end{adjustbox}
		\end{table}
		
		Cuerpo de la petición (\textit{JSON}):
		
		\begin{table}[h!]
			\centering
	\begin{adjustbox}{max width=\textwidth}
			\begin{tabular}{|l|l|l|l|}
				\hline
				Parámetro & Tipo & Opcional & Descripción \\ \hline
				operation & string & No & Nombre de la operación que se quiere ejecutar. Valores posibles: \textit{list}, \textit{get}, \textit{prune}, \textit{pull}, \textit{remove}, \textit{search}. \\ \hline
				data & dict & No & Argumentos de la operación. \\ \hline
			\end{tabular}
\end{adjustbox}
		\end{table}
	
	\pagebreak
		Tipos de operaciones:
		
			\subsubsection{list}
				\begin{table}[h!]
					\centering
	\begin{adjustbox}{max width=\textwidth}
					\begin{tabular}{|l|l|l|l|l|}
						\hline
						Parámetro & Key & Tipo & Opcional & Descripción \\ \hline
						data &  & dict & No & Argumentos de la operación. \\ \hline
						& name & string & Sí & Mostrar sólo las imágenes pertenecientes a este repositorio. \\ \hline
						& all & bool & Sí & Mostrar todas las imágenes o no (incluídas las imágenes de capas intermedias). \\ \hline
						& filters & \hyperref[sec:filtroimagen]{Filtro} & Sí & Filtros adicionales. \\ \hline
					\end{tabular}
\end{adjustbox}
				\end{table}
			
				Respuesta (una sola imagen):
				\begin{table}[h!]
					\centering
	\begin{adjustbox}{max width=\textwidth}
					\begin{tabular}{|l|l|l|}
						\hline
						Parámetro & Tipo & Descripción \\ \hline
						data & \hyperref[sec:imagen]{Imagen} & Diccionario con los datos de la imagen. \\ \hline
					\end{tabular}
\end{adjustbox}
				\end{table}
			
				Respuesta (más de una imagen):
				\begin{table}[h!]
					\centering
	\begin{adjustbox}{max width=\textwidth}
					\begin{tabular}{|l|l|l|}
						\hline
						Parámetro & Tipo & Descripción \\ \hline
						total & int & Número de imágenes listadas. \\ \hline
						items & list[\hyperref[sec:imagen]{Imagen}] & Imágenes listadas. \\ \hline
					\end{tabular}
\end{adjustbox}
				\end{table}
			
			\subsubsection{get}
				\begin{table}[h!]
					\centering
	\begin{adjustbox}{max width=\textwidth}
					\begin{tabular}{|l|l|l|l|l|}
						\hline
						Parámetro & Key & Tipo & Opcional & Descripción \\ \hline
						data &  & dict & No & Argumentos de la operación. \\ \hline
						& name & string & No & Nombre de la imagen. \\ \hline
					\end{tabular}
\end{adjustbox}
				\end{table}
			
				Respuesta: \hyperref[sec:imagen]{Imagen}
				
			\subsubsection{prune}
				\begin{table}[h!]
					\centering
	\begin{adjustbox}{max width=\textwidth}
					\begin{tabular}{|l|l|l|l|l|}
						\hline
						Parámetro & Key & Tipo & Opcional & Descripción \\ \hline
						data &  & dict & No & Argumentos de la operación. \\ \hline
						& filters & \hyperref[sec:filtroimagen]{Filtro} & Sí & Filtros adicionales. \\ \hline
					\end{tabular}
\end{adjustbox}
				\end{table}
			
			\pagebreak
			\subsubsection{pull}
				\begin{table}[h!]
					\centering
	\begin{adjustbox}{max width=\textwidth}
					\begin{tabular}{|l|l|l|l|l|}
						\hline
						Parámetro & Key & Tipo & Opcional & Descripción \\ \hline
						data &  & dict & No & Argumentos de la operación. \\ \hline
						& repository & string & Sí & Repositorio e imagen a descargar. \\ \hline
						& tag & string & Sí & Tag de la imagen. \\ \hline
						& auth\_config & dict & Sí & Sobreescribir las credenciales. \\ \hline
						& platform & string & Sí & Plataforma en formato: os[/arch[/variant]] \\ \hline
					\end{tabular}
\end{adjustbox}
				\end{table}
			
				Respuesta: \hyperref[sec:imagen]{Imagen} descargada.
				
			\subsubsection{remove}
				\begin{table}[h!]
					\centering
	\begin{adjustbox}{max width=\textwidth}
					\begin{tabular}{|l|l|l|l|l|}
						\hline
						Parámetro & Key & Tipo & Opcional & Descripción \\ \hline
						data &  & dict & No & Argumentos de la operación. \\ \hline
						& image & string & No & Imagen a eliminar. \\ \hline
						& force & bool & Sí & Forzar borrado. \\ \hline
						& noprune & bool & Sí & Borrar o no imágenes padre sin tag. \\ \hline
					\end{tabular}
\end{adjustbox}
				\end{table}
			
			\subsubsection{search}
				\begin{table}[h!]
					\centering
	\begin{adjustbox}{max width=\textwidth}
					\begin{tabular}{|l|l|l|l|l|}
						\hline
						Parámetro & Key & Tipo & Opcional & Descripción \\ \hline
						data &  & dict & No & Argumentos de la operación. \\ \hline
						& term & string & No & Término de búsqueda. \\ \hline
					\end{tabular}
\end{adjustbox}
				\end{table}
			
				Respuesta:
				
				\begin{table}[h!]
					\centering
	\begin{adjustbox}{max width=\textwidth}
					\begin{tabular}{|l|l|l|l|}
						\hline
						Parámetro & Key & Tipo & Descripción \\ \hline
						total &  & int & Número de imágenes encontradas. \\ \hline
						items &  & list[dict] & Imágenes encontradas. \\ \hline
						& star\_count & int & Número de estrellas en DockerHub. \\ \hline
						& is\_official & bool & Si la imagen es oficial o no. \\ \hline
						& name & string & Nombre de la imagen \\ \hline
						& is\_automated & bool & Imagen automatizada. \\ \hline
						& description & string & Descripción. \\ \hline
					\end{tabular}
\end{adjustbox}
				\end{table}
		
	
	
	\pagebreak
	\subsection{\textit{POST /deploy/image/single}}
		Permite ejecutar operaciones básicas en una imagen en concreto.
		
		Cabeceras necesarias:
		\begin{table}[h!]
			\centering
	\begin{adjustbox}{max width=\textwidth}
			\begin{tabular}{|l|l|l|}
				\hline
				Nombre & Opcional & Descripción \\ \hline
				x-access-token & No & Token de acceso. \\ \hline
			\end{tabular}
\end{adjustbox}
		\end{table}
		
		Cuerpo de la petición (\textit{JSON}):
		
		\begin{table}[h!]
			\centering
	\begin{adjustbox}{max width=\textwidth}
			\begin{tabular}{|l|l|l|l|}
				\hline
				Parámetro & Tipo & Opcional & Descripción \\ \hline
				name & string & No & Nombre de la imagen. \\ \hline
				operation & string & No & Nombre de la operación que se quiere ejecutar. Valores posibles: \textit{history}, \textit{reload}. \\ \hline
				data & dict & No & Argumentos de la operación. \\ \hline
			\end{tabular}
\end{adjustbox}
		\end{table}
	
		Tipos de operaciones:
		
			\subsubsection{history}
				Respuesta:
				
				\begin{table}[h!]
					\centering
	\begin{adjustbox}{max width=\textwidth}
					\begin{tabular}{|l|l|l|}
						\hline
						Parámetro & Tipo & Descripción \\ \hline
						data & string & Historia de la imagen. \\ \hline
					\end{tabular}
\end{adjustbox}
				\end{table}
			
			\subsubsection{reload}
				Recarga la imagen y aplica los posibles cambios que tenga.
		
		
	




	\chapter{Variables de entorno}

\textbf{IPManager} necesita algunas variables de entorno para funcionar correctamente. Todas tienen un valor por defecto, aunque se recomienda que se revisen y se modifiquen antes de comenzar a usar tanto backend como frontend.

\section{Backend}

Estas variables se pueden configurar como variables de entorno o se pueden configurar directamente en \textsf{config/server\_environment.py}.

\begin{itemize}
	\item \textsf{MONGO\_HOSTNAME} Hostname donde se encuentra el host de MongoDB. Por defecto \textsf{127.0.0.1}.
	\item \textsf{MONGO\_PORT} Puerto del host de \textit{MongoDB}. Por defecto \textsf{27017}.
	\item \textsf{TESTING\_DATABASE} Nombre de la colección usada para ejecutar los tests unitarios. Por defecto \textsf{ipm\_root\_testing}.
	\item \textsf{BASE\_DATABASE} Nombre de la colección base de IPManager. En esta colección se almacenan datos de los clientes del backend. Por defecto toma el valor que tenga la variable \textsf{TESTING\_COLLECTION}.
	\item \textsf{ENC\_KEY} Clave usada para encriptar las contraseñas. Puedes generar una clave ejecutando el archivo \textsf{generate\_key.py} que se encuentra en \textsf{utils}. Por defecto se usa una aleatoria.
	\item \textsf{JWT\_ENC\_KEY} Clave usada para encriptar los \textit{JWT} usados en el login. Puedes generar una clave ejecutando el archivo \textsf{generate\_key.py} que se encuentra en \textsf{utils}. Por defecto se usa una aleatoria.
	\item \textsf{DOCKER\_BASE\_URL} URL o path donde se encuentra el socket de Docker. Por defecto \textsf{unix://var/run/docker.sock}.
	\item \textsf{DOCKER\_ENABLED} Activa o desactiva los endpoints para gestión de servicios. Por defecto comprueba si el backend se está ejecutando en un Docker para desactivarlos en caso afirmativo.
	\item \textsf{ANSIBLE\_PATH} Path relativo o absoluto donde se van a almacenar los diferentes archivos generados por el backend necesarios para el aprovisionamiento. Por defecto \textsf{./}.
\end{itemize}


\section{Frontend}

El frontend cuenta con dos variables de entorno que se tienen que configurar previamente, son:

\begin{itemize}
	\item \textsf{backendUrl}. URL del backend sin protocolo.
	\item \textsf{httpsEnabled}. \textit{True} si el backend cuenta con HTTPS, \textit{false} en caso contrario.
\end{itemize}

La configuración de estas variables se puede hacer de dos maneras:

\begin{itemize}
	\item En los archivos \textsf{environment.*.ts} que se encuentran en \textsf{fronend/src/environments}. Estos archivos una vez configurados son inmutables una vez se ha construido el frontend.

	\begin{itemize}
		\item \textsf{environment.on-premise.ts}. Entorno utilizado para construir la imagen de Docker. Por defecto se utilizan los valores usados en el \textit{docker-compose}.
		\item \textsf{environment.prod.ts}. Entorno utilizado para construir una versión de producción.
	\end{itemize}

	\item En el archivo \textsf{env.js}. Este archivo se encuentra en \textsf{fronend/src} y se puede modificar sin tener que reconstruir el frontend. En caso de no necesitar esta característica se recomienda utilizar el environment anterior. El archivo tiene la siguiente forma y se deben descomentar las líneas 2 a 5 para su uso.
	
\begin{lstlisting}
(function (window) {
// window.__env = {
//   backendUrl: '172.20.0.3:5000',
//   httpsEnabled: false
// };
}(this));
\end{lstlisting}
\end{itemize}

	\chapter{Instalación del sistema}

Una vez clonado el repositorio se puede instalar y ejecutar tanto backend como frontend siguiendo los pasos que se describen a continuación.

\section{Backend}
El backend de \textbf{IPManager} tiene algunas dependencias que se tienen que instalar para que funcione correctamente, son las siguientes:

\begin{lstlisting}
apt install sshpass
apt instal openssl
apt install libffi6

pip3 install -r requirements.txt
\end{lstlisting}

La versión instalada del paquete \textit{werkzeug} debe ser 0.16.1.

Se puede ejecutar con \textit{Flask} o \textit{Gunicorn}.
\begin{itemize}
	\item Flask (no recomendado):
\begin{lstlisting}
export FLASK_APP=wsgi.py

flask run
\end{lstlisting}
	
	\item Gunicorn:
\begin{lstlisting}
gunicorn -b 0.0.0.0:5000 wsgi:app
\end{lstlisting}
	
\end{itemize}

\subsection{\textit{Docker}}
El backend también está disponible en \textit{Docker}, la imagen puede descargarse de la siguiente manera:

\begin{lstlisting}
docker pull harvestcore/ipm-backend:<tag>
\end{lstlisting}

Se recomienda siempre utilizar la última versión disponible de la imagen, la cual puede consultarse \href{https://github.com/harvestcore/tfg/releases}{aquí} o \href{https://hub.docker.com/r/harvestcore/ipm-backend/tags}{aquí}. Deben comprobarse también las \hyperref[sec:variables]{variables de entorno} necesarias para ejecutar el backend.

Ejemplo de ejecución:

\begin{lstlisting}
docker run -e MONGO_HOSTNAME=172.20.0.2 harvestcore/ipm-backend:<tag>
\end{lstlisting}

En el caso de querer construir la imagen se debe ejecutar:

\begin{lstlisting}
cd backend

docker build . -t ipm-backend:<tag>
\end{lstlisting}





\section{Frontend}

Para instalar el frontend se deben revisar y configurar las \hyperref[sec:variables]{variables de entorno}, tras eso se debe ejecutar lo siguiente:

\begin{lstlisting}
cd frontend

npm build --prod
\end{lstlisting}

Para ejecutarlo se recomienda utilizar \textit{Nginx} u otro tipo de servidor web. En la raíz del frontend se adjunta el archivo de configuración (\textit{nginx.conf}) usado para construir la imagen de \textit{Docker}, y tambien puede ser usado en este caso.



\subsection{\textit{Docker}}

Se puede ejecutar el frontend con \textit{Docker}, para ello se puede descargar la imagen del repositorio disponible o se puede construir de forma local.

\begin{lstlisting}
docker pull harvestcore/ipm-frontend:<tag>
\end{lstlisting}

El tag o versión se puede consultar \href{https://github.com/harvestcore/tfg/releases}{aquí} o \href{https://hub.docker.com/r/harvestcore/ipm-frontend/tags}{aquí}. Se recomienda usar siempre la última versión estable.

\begin{lstlisting}
cd frontend

// Construir imagen
docker build . -t ipm-frontend:<tag>

docker run ipm-frontend:<tag>
\end{lstlisting}



\section{Docker-compose}

En el caso de utilizar el \textit{docker-compose} que se encuentra en la raíz del repositorio solo es necesario ejecutar lo siguiente:

\begin{lstlisting}
docker-compose build

docker-compose up
\end{lstlisting}

Por supuesto se pueden agregar variables de entorno para configurar el backend. Un ejemplo sería:

\begin{lstlisting}
docker-compose up -e BASE_DATABASE=ipm_root
\end{lstlisting}

El docker-compose tiene configurada una red \textit{bridge} con la siguiente \textit{subnet}:

\begin{itemize}
	\item 172.20.0.0/16
\end{itemize}

Por otro lado las máquinas cuentan con las siguientes direcciones IP estáticas asignadas:

\begin{itemize}
	\item mongo: 172.20.0.2
	\item ipmanager-backend: 172.20.0.3
	\item ipmanager-frontend: 172.20.0.4
\end{itemize}

También se fija la variable de entorno \textit{BASE\_DATABASE} con valor \textit{ipm\_root}.
	\chapter{Primeros pasos y \textit{CLI}}

\section{Inicialización de la base de datos}
Tras instalar el backend debes ejecutar un script llamado \textsf{init\_database.py} (se encuentra en la raíz del proyecto), el cual creará un usuario en el cliente base. Este usuario es administrador y sus credenciales deben ser cambiadas una vez haya sido creado.

El cliente base viene denominado por el nombre de la base de datos principal, el cual se toma de la variable de entorno \textsf{BASE\_DATABASE} (toma valor \textsf{ipm\_root} en caso de no encontrarse configurada).

Tras crear este primer usuario administrador, este puede comenzar a crear otros usuarios o clientes usando la \textit{API} o el \textit{CLI}.

\section{\textit{CLI}}

El \textit{CLI} que se incluye en el directorio raíz del backend permite realizar algunas operaciones con clientes (o \textit{customers}) y usuarios. Se puede ejecutar de la siguiente manera:

\begin{lstlisting}
cd backend

python3 cli.py
\end{lstlisting}

Es interactivo y permite:

\begin{itemize}
	\item Crear un cliente.
	\item Activar un cliente.
	\item Desactivar un cliente.
	\item Agregar un usuario a un cliente.
\end{itemize}

Internamente utiliza la configuración de las \hyperref[sec:variables]{variables de entorno} que se encuentre establecida en el momento de utilizar el \textit{CLI}.
	%\input{glosario/entradas_glosario}
	% \addcontentsline{toc}{chapter}{Glosario}
	% \printglossary

	\nocite{*}
	\bibliographystyle{plain}
	\bibliography{bibliografia/bibliografia}


\end{document}
